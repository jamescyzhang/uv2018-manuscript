% !TeX spellcheck = en_US
\documentclass[letterpaper, twocolumn, 10pt, conference]{IEEEtran}
\IEEEoverridecommandlockouts
\usepackage[utf8]{inputenc}
\usepackage[T1]{fontenc}
\usepackage{geometry}
\usepackage{amsmath}
\usepackage{amsfonts}
\usepackage{amssymb}
\usepackage{indentfirst}
\usepackage{textcomp}
\usepackage{lipsum}
\usepackage{contour}
\usepackage[normalem]{ulem}
\usepackage{graphicx}
\usepackage{float}
%\usepackage{nohyperref}
\usepackage{multirow}
\usepackage{tabularx}
\usepackage{todonotes}
\usepackage{perpage}
\usepackage{url}
\usepackage{csquotes}
\usepackage{pdflscape}
\usepackage{subcaption}
\usepackage[style=ieee, backend=bibtex]{biblatex}
\usepackage{array}
\newcolumntype{P}[1]{>{\centering\arraybackslash}m{#1}}

\geometry{letterpaper, top=1in, bottom=1in, left=18mm, right=18mm}

\setlength{\marginparwidth}{1.5cm}
\setlength{\columnsep}{5mm}
\setlength{\parindent}{3.5mm}

\newcolumntype{Y}{>{\centering\arraybackslash}X}

\newcommand{\term}[1]{\textbf{#1}}
\newcommand{\mono}[1]{\texttt{#1}}


\addbibresource{reference_FuxinDu.bib}
\addbibresource{reference_GuanghuaCheng.bib}
\addbibresource{reference_JamesZhang.bib}
\addbibresource{reference_JieZheng.bib}
\addbibresource{reference_LongfeiZhou.bib}
\addbibresource{reference_ShengshengCao.bib}
\addbibresource{reference_WenyangGao.bib}
\addbibresource{reference_YajunFang.bib}
\addbibresource{reference_YanxiChen.bib}
\addbibresource{reference_ZejunZhang.bib}
\addbibresource{reference_GuanghuaCheng_ex.bib}


\begin{document}
        
\title{Preliminary Study on Evaluation of Smart-Cities Technologies and Proposed UV Lifestyles}

\author{
	\IEEEauthorblockN{Shengsheng Cao}
	\IEEEauthorblockA{
		\textit{Tsinghua University}\\
		Beijing, China\\
		05d\_cao@163.com}
	\and
	\IEEEauthorblockN{Yanxi Chen}
	\IEEEauthorblockA{
		\textit{Massachusetts Institute of Technology}\\
		Boston, MA, USA \\
		ychen745gt@gmail.com}
	\and
	\IEEEauthorblockN{Guanghua Cheng}
	\IEEEauthorblockA{
		\textit{Northeastern University}\\
		Boston, MA, USA\\
		cheng.gu@husky.neu.edu}
	\and
	\IEEEauthorblockN{Fuxin Du}
	\IEEEauthorblockA{
		\textit{Shanghai Normal University}\\
		Shanghai, China \\
		dufuxin.francie@gmail.com}
	\and
	\IEEEauthorblockN{Wenyang Gao}
	\IEEEauthorblockA{
		\textit{Sichuan University}\\
		Chengdu, China \\
		wygao8@gmail.com}
	\and
	\IEEEauthorblockN{Haoran Ma}
	\IEEEauthorblockA{
		\textit{Jilin University}\\
		Jilin, China \\
		841615728qq@gmail.com}
	\and
	\IEEEauthorblockN{Qikai Su}
	\IEEEauthorblockA{
		\textit{University of Nottingham}\\
		Nottingham, United Kingdom \\
		suqikai@outlook.com}
	\and
	\IEEEauthorblockN{Chuyuan Zhang}
	\IEEEauthorblockA{
		\textit{Univ. of California, Los Angeles}\\
		Los Angeles, CA, USA\\
		james.cy.zhang@gmail.com}
	\and
	\IEEEauthorblockN{Zejun Zhang}
	\IEEEauthorblockA{
		\textit{Wuhan University}\\
		Wuhan, China \\
		735256945zzj@gmail.com}
	\and
	\IEEEauthorblockN{Jie Zheng}
	\IEEEauthorblockA{
		\textit{Beijing Univ. of Posts and Telecomm.}\\
		Beijing, China \\
		jiezheng@bupt.edu.cn}
	\and
	\IEEEauthorblockN{Longfei Zhou}
	\IEEEauthorblockA{
		\textit{Beihang University}\\
		Beijing, China \\
		zhoulf999@gmail.com}	
	\and
	\IEEEauthorblockN{Yajun Fang\IEEEauthorrefmark{1}}\thanks{\IEEEauthorrefmark{1} Corresponding author. Other authors are listed in alphabetical order.}
	\IEEEauthorblockA{
		\textit{Mass. Institute of Technology}\\
		Boston, MA, USA \\
		yjfang@mit.edu\\
	}
}

\IEEEpubid{\makebox[\columnwidth]{978-1-5386-5197-1/0/\$31.00˜\copyright˜2018 IEEE \hfill } 
\hspace{\columnsep}\makebox[\columnwidth]{\hfill }}

\maketitle

\begin{abstract}

Our current society is facing challenges in both sustainability and environmental pollution due to fast urbanization, limited resources and increasing senior population. Smart cities which aims to increase efficiency and convenience would not be able to solve fundamental challenges caused by urban life styles. In 2013, the Universal Village concept was proposed to enhance human-nature harmony through prudent use of technologies and to address the eco-challenges due to fast urbanization. 

This paper first studies the environmental implications due to urban lifestyles and proposes the suitable UV framework and detailed content of universal village lifestyle in order to address the eco-challenges. The paper then evaluates the development of current smart city technologies and assesses their validity with regard to thpt of Universal Village through systematic studies of several major intelligent systems.

Specifically, this paper discusses the subject of connectivity from four perspectives: feedback loop, mutual interaction, dynamic information loop, and material cycle. The paper evaluates whether information feedback loops could be formed for these major systems, and also explores the mutual interaction and dependence among the seemingly independent major systems. We discover that mutual interaction connects the aforementioned systems into an interconnected network and naturally forms dynamic information loops in which the decision of one system may be the required input of another system, or vice versa. This implies that proper functioning of these systems requires extensive information sharing among them. One event might dynamically trigger different events. The last connectivity is material cycle. 
We explore the whole life cycle of products, including impact from lifestyle, customers’ need, product design, cloud manufacturing, sale channel, feedback collection from customers, reuse and recycling, scrapping, to final waste-disposal, etc., and study how to reduce the demand for resource and waste during the procedure. 
The idea is to include the perspective of UV lifestyle when designing product, consider the possibility in proactively reducing need, sharing product with different people, reusing product parts into the manufacturing, recycle re-usable components of finished products before the products’ being fully disassembled, etc. The advantage is to reduce the need for products, and to avoid manufacturing the same components from raw materials directly, which demands less resource. 

In summary, connectivity as discussed from the four perspectives would greatly contribute to the effectiveness and efficiency of our connected smart systems. Dynamic information loop helps coordinate resource allocation, decreases the collective costs, and reduces demand of natural resources from natural environment, resulting in less damage to environment which ultimately enhances system-wide harmony between human and its natural environment, and leads to human happiness in general.

\end{abstract}

\begin{IEEEkeywords}
	
Universal Village, Smart Cities, Human-Nature Harmony, Top-Down Design, System Theory, System Dynamics, Connectivity, Mutual Interaction, Mobility, Preventive Healthcare, Well-being, Proactive, Economic Driven, Intelligent Healthcare, Intelligent Transportation Systems, Intelligent Environment

\end{IEEEkeywords}

\section{Introduction: Urbanization, Smart Cities and Challenges}
\label{sec:Introduction}

Our contemporary society faces challenges due to fast urbanization, limited resources and increasing population of senior citizens. The concept of smart cities has been widely adopted in order to address the problems. However, many current designs are availability-based, technique-oriented, and bottom-up schemes which focus on individual elements and could not address the Eco-challenges.

\subsection{Urbanization and Growing Markets for Smart Cities}
\label{ssec:growing_markets}

In the recent few years, urbanization has becoming a worldwide trend as shown in Fig.~\ref{fig:fyj:urbanization_trend}. The ratio of city population will increase from 50\% to 70\% in the world. The rapid urbanization makes it hard for new residents to access city resources \& public services and poses new challenges, including crowdedness, traffic, energy, pollution, etc.

\begin{figure}[h!]
        \centering
        \includegraphics[width=0.6\linewidth]{images/fyj-urbanization_trend}
        \caption{The development trend of urbanization}
        \label{fig:fyj:urbanization_trend}
\end{figure}

The concept of smart cities has been widely adopted in order to address the eco-challenges from different perspectives, including intelligent security monitoring, intelligent healthcare, intelligent transportation systems, etc., as in Fig.~\ref{fig:fyj:smart_city}. Many emerging technologies are being developed for the fundamental infrastructure of future societies, such as IoT, cloud computing, etc. The market of such intelligent systems is rapidly increasing as shown in Fig.~~\ref{fig:fyj:iot_market} and Fig.~\ref{fig:fyj:ITS_market}. AI/ICT technologies are seen as driving factors for future economics and the solutions for current challenges.

\begin{figure}[h!]
    \centering
    \includegraphics[width=0.9\linewidth]{images/fyj-smart_city}
    \caption{Hierarchical Structure for Smart Cities}
    \label{fig:fyj:smart_city}
\end{figure}

\begin{figure}[h!]
	\centering
	\begin{subfigure}[h!]{0.45\linewidth}
	    \includegraphics[width=\linewidth]{images/fyj-iot_market}
	    \caption{IoT market}
	    \label{fig:fyj:iot_market}
	\end{subfigure}
	\begin{subfigure}[h!]{0.45\linewidth}
		\includegraphics[width=\linewidth]{images/fyj-ITS_market}
		\caption{ITS market}
		\label{fig:fyj:ITS_market}
	\end{subfigure}
	\caption{The growing trend for IoT and ITS markets}
\end{figure}

\subsection{Rapid Trend of Urbanization Poses Eco-Challenges to Resources and Services}
\label{ssec:urbanization-challenges}

However, these technologies fail to address the fundamental eco-challenges for urbanization: high GDP per capita comes at the cost of high metabolic rates and high waste per capita, respectively shown in Fig.~\ref{fig:fyj:gdp_resource} and Fig.~\ref{fig:fyj:waste_country}~\cite{UN-world-energy-report}. 
People might not realize that urban lifestyle comes at such a high cost. As demonstrated in Fig.~\ref{fig:fyj:gdp_resource} and Fig.~\ref{fig:fyj:waste_country}, individuals' demand for resources and the amount of waste production in developed countries are several times higher than that in developing countries. 
The urbanization trend leads to changes in living style \& industrialization as well as higher demands in nature resources \& waste production. Thus, urbanization is actually bringing regions/countries with low demand \& low waste-disposal to urban lifestyle with high demand \& high waste-disposal and thus leads to over-exploitation of nature and serious sustainability issues, environmental pollution, energy crisis, global warming, etc.
If more countries move from the bottom left corner to the top right corner in Fig.~\ref{fig:fyj:gdp_resource}, the speed of draining our limited nature resources will be even faster if we do not change our current way of living and manufacturing, and the waste-disposal will be increased at much higher speed, causing irreversible environmental damages.  

The correlation among high resource usage, large amount of waste, and income is provided by the report from United Nations Environment Program (UNEP)~\cite{united2011decoupling}.  The UNEP report warns that \enquote{by 2050, humanity could devour an estimated 140 billion tons of minerals, ores, fossil fuels and biomass per year – three times its current appetite – unless the economic growth rate is \enquote{decoupled} from the rate of natural resource consumption.} 


\begin{figure}[h!]
        \centering
        \includegraphics[width=0.8\linewidth]{images/fyj-gdp_resource}
        \caption{Global Interrelation between Resource Use and Income (175 Economies in Year 2000)}
        \label{fig:fyj:gdp_resource}
\end{figure}





\begin{figure}[h!]
        \centering
        \includegraphics[width=0.7\linewidth]{images/fyj-waste_by_country}
        \caption{Waste generation per capita (kg/day) to gross income (GNI)}
        \label{fig:fyj:waste_country}
\end{figure}




\subsection{The Most Expensive Lifestyle does not Guarantee Satisfying Safety and Healthcare}
\label{ssec:expensive_lifestyle}

The nations at the right/upper corners in Fig.~\ref{fig:fyj:gdp_resource} with their expensive lifestyle also run into significant challenges in both safety and healthcare as discussed in paper~\cite{mit-uv2014}, section 3-C. 

Firstly, the lifestyle on the wheel is not safe. 
% Traffic accident is anticipated to be the third most prevalent cause of death worldwide by 2020. 
% In US, the fatality rate for seniors is 17 times as high as for people of age 25-65. 
According to World Health Organization, annual traffic fatalities and injuries are around 1.35M and 50M. UN report sets the goal of decreasing the total number of deaths / injuries in half by 2020, as mentioned in the 2030 Agenda for Sustainable Development~\cite{fatal-accident}.

In addition, the healthcare systems become too expensive to be sustainable even for their own residents due to complicated diseases and chronic diseases. The average health level might even degrade in some regions due to the unhealthy living habits as well as the high stress. As shown in Fig.~\ref{fig:fyj:US-healthcost-lifespan-2017}, the total expenditure has been more than tripled in order to extend the average life expectancy from 75 to 80 years. Healthcare situation for the United States is far below the general trend. Average healthcare cost in the US, 9000 dollars per capita, is more than twice the cost in the United Kingdom, while the average life expectancy in the US is lower than that in the UK. 

\begin{figure}[h!]
        \centering
        \includegraphics[width=0.8\linewidth]{images/fyj-US-healthcost-lifespan-2017}
        \caption{Global Interrelation between Resource Use and Income (175 Economies in Year 2000)}
        \label{fig:fyj:US-healthcost-lifespan-2017}
\end{figure}

\subsection{Development Goals from United Nations}
\label{ssec:UN-development-goals}

The United Nations (UN) proposed Sustainable Development Goals (SDG) for 2015–2030~\cite{ UN-sustainable-development} and includes an urban goal: \enquote{inclusive, safe, resilient, and sustainable cities} with details on targets and 14 indicators.

While the current technologies for smart cities aim for efficiencies, they do not promise cities to be more \enquote{inclusive, safe, resilient, and sustainable} as proposed by the UN. Some people claim that \enquote{Smart Cities are seen to be essentially a branding war between different multinational corporations in the information, communication, and technological (ICT)}~\cite{allam2018redefining}.  Specifically, as discussed in section~\ref{sec:SmartTechLimitation} and~\ref{ssec:AI.limitation}, the current smart city or AI-based agent might make people’s experience even worse than before. 

\subsection{Proposed Solution: Universal-Village Lifestyle}
\label{ssec:UV.solution}

To address the aforementioned challenges, in 2013, MIT Universal Village (UV) Program puts forward our vision for future society and community development, \enquote{Universal Village}, an upgraded version of Smart Cities in which people and nature are in harmony through the wise application of advanced technology. 
The word, Universal, symbolizes that human should follow the laws of the universe and protect our environment and ecosystems. In 2014, the concept of Universal Village(UV) Lifestyle was officially recommended by MIT UV Program to address the eco-challenges due to urbanization~\cite{mit-uv2014}, to improve quality of life, and to provide sustainable happiness for the future of humankind. 

This paper is to report our follow-up studies on the framework of Universal Village as well as on the basic elements of proposed UV lifestyle. We first discuss the limitations of general smart technologies in section~\ref{sec:SmartTechLimitation} to demonstrate the necessity and effectiveness of Universal Village framework. 
Then we propose ideal UV lifestyles in section~\ref{sec:UVConcept} and~\ref{sec:Information_Dissemination}, and conduct in-depth comparison between UV standard and corresponding subsystem of Smart Cities in section~\ref{sec:Subsystems}, which is to raise people’s awareness and to showcase the possible directions and solutions to the current challenges. 
% The comparison between the ideal UV lifestyles and the current status is expected to raise people’s awareness and to showcase the possible directions and solutions to our challenges.

\section{The Limitation for Current Smart technologies and Smart Cities}
\label{sec:SmartTechLimitation}

The development of smart cities, despite the growing market, has been limited to the application in the information, communication, and technology (ICT) domains.~\cite{allam2018redefining}. Many current designs are availability-based, technique-oriented, and bottom-up schemes which focus on individual elements and might even fail to achieve their efficiency goals, which will be discussed below. 
        
%YF: I switched the order of two following subsubsections{Incomplete Feedback Loop Control} and {Uncoordinated Subsystems}

 
\subsection{Uncoordinated Subsystems}
        
The first paper to have proposed Universal Village concept~\cite{mit-uv2013} thoroughly discussed the importance of system design. At this moment individual elements for Smart Cities are still designed independently without considering the potential interaction among these elements as shown in Fig.~\ref{fig:fyj:subapp_interaction}.
        
\begin{figure}[h!]
    \centering
    \includegraphics[width=0.8\linewidth]{images/fyj-subapp_interaction}
    \caption{The potential interaction among multiple intelligent systems}
    \label{fig:fyj:subapp_interaction}

\end{figure}
%Global Interrelation between Resource Use and Income (175 Economies in Year 2000)

%\subsubsection{Connected Information and Incomplete Feedback Loop Control}
\subsection{Incomplete Feedback Loop Control}

        
At this moment, most intelligent subsystems for Smart Cities are still at the stage of data acquisition and information connection, and they cannot actually take advantage of these data to aid decision and implement real-time response system. If the information from the feedback loop could not be used, the feedback loop control would not be feasible. Tons of data are collected and recorded mainly for the purpose of post-processing if there is any unexpected urgency. 
        
                
\subsection{Limitation of Learning-based Systems and their Social Implications}
\label{ssec:AI.limitation}
                
For a long period, there have been debates between knowledge-based learning methods and statistics-based learning methods, among which each has its own advantages and disadvantages. 

Knowledge-based logistic learning methods present a good interpretability. The limitations of such systems mainly lie upon their feature engineering tasks. Knowledge and rule sets are sensitive to scenarios, for which reasoning systems must be changed accordingly. Such disadvantages hinder a knowledge-based machine learning methods in the application level.
                
In comparison, statistics-based machine learning technologies have achieved breakthroughs thanks to a boost in computational power. Statistics-based methods make little assumptions on their samples, and iteratively approach to real distributions by processing large amount of data, hence achieving compatibility and adaptability in different tasks. Their high correlation to big data techniques, however, is not only a boon but also a curse; applicable models must identify between training and application populations, which in turn put requirements on data acquisition. 
                
\begin{itemize}
\item Inclusive vs. Experiences for Outliers
	
The rapid development of AI-based system is gradually taking the place of humans’ operation such as big data analysis and face recognition. While models homogenize individuals, minority on the amount of data are paying penalty for outlier behaviors off the main distribution. AI-based systems respond poorly to any non-routine behaviors or requests. To make it even worse, it is possible for such systems to classify people’s normal behavior as outliers since the current AI systems fail in understanding common sense. Users or customers have already experienced the pain and frustration in many intelligent systems, AI-based customer services, security management, etc., an experience that goes contrary to  the UN development goals, inclusivity, discussed in section~\ref{ssec:UN-development-goals}. 



\item Impact of Lacking Interpretability

%	Last but not the least, lack of interpretability obstructs model applications in many scenarios. 
Lack of interpretability obstructs model applications in many scenarios. 
In applications, limited model understanding can result in distrust and severe moral crisis. From a correct distribution, wrong correlations get drawn between characteristics for a lack of reasoning. 
                                                
\item New Segregation \& Echo Chamber Effort due to Personalized Service and Erroneous Profiling
                        
% Many intelligent systems provide personalized service based on users’ profiling and start to provide target services and information dedicated to that group. While users access the same system, their information and experiences might differ significantly, which many users and customers have not realized yet.
Many intelligent systems start to identify users’ profile and to provide target or personalized services and information dedicated to different user groups. Many users and customers have not realized yet that they are offered different information and experiences when accessing the same system. 
Such \enquote{personalized} services create new segregation and make it more difficult for people to understand each other. 
                        
Due to the limitation of current AI systems, users’ random behaviors can also be wrongfully profiled as patterns. Users get trapped in an erroneous profiling due to the \enquote{Echo Chamber Effect}, and consequently get fed with biased information which in turn prompts a wrong cognition. Thus Vicious loops are formed. 
% Users get trapped in an erroneous profiling due to the \enquote{Echo Chamber Effect}; vicious loops are formed, as biased information get fed on users which in turn prompts a wrong cognition.
                        
\item Privacy, Safety, Bias and Discrimination Concern
                        
While the function of systems depends on the reliability of data and training sample, the whole intelligent systems are subject to attack or failure due to privacy leak, data theft and sensor failure. 
%Lacking in standard and data bias, erroneous profiling and labeling, might introduce unnecessary hostile attitude or discrimination toward their users. 
The data bias, lack of data standard, and erroneous profiling / labeling, might introduce unnecessary hostile attitude or discrimination toward their users. 

                        
Besides, for intelligent systems, the privacy leaks normally disclose dynamic information on individuals or their home, which would cause more serious damage than static privacy information. 
                        
\item Hybrid AI and Human System
                        
With its multiple limitations, machine learning is neither an elixir nor a universal standard. Relying solely on AI to support the operation of smart cities will possibly raise serious problems. 
In a smart city scenario, pure AI-based agents or applications must undergo rigorous designs and careful examinations before taking over responsibilities.
%In a smart city scenario, artificial intelligence applications must undergo rigorous designs and careful examinations before delegating more function to pure AI-based agents. 

An efficient smart city in the future needs to combine AI and human beings interdependently, and make them work cooperatively. Human beings have some abilities that artificial intelligence does not have, such as the understanding of common sense, imagination, creativity and comprehensive judgment. 
Specifically, such hybrid system is very important for those people who do not have access to smart devices or do not know how to use smart devices. Otherwise they would feel abandoned by the \enquote{smart} society. 
Only in this way can the whole smart city system be more humanized, more inclusive, more diversified, more resilient, and more concerned with special situations rather than mechanical judgment and classification.                 

\end{itemize}

\section{UV Concept and System Framework} 
\label{sec:UVConcept}
                
As introduced in section~\ref{ssec:UV.solution}, MIT proposed the concept of \enquote{Universal Village} (UV), an upgraded version of Smart Cities in which people and nature are in harmony through the wise application of advanced technology. 
Before we proceed to introduce the detail elements of UV framework, we will elaborate on the essence of UV concept: what is so-called \enquote{sustainable happiness} and \enquote{harmony between man and nature}?  How is it presented in our daily lives? 
How do we achieve the UN SDG goal of  \enquote{inclusive, safe, resilient, and sustainable}?
This is the theoretical premise that we must focus on in order to understand the UV concept. This will also become an important value system and concept guide for us to design various intelligent systems and technology exploration directions.

\subsection{Merit of UV Concept}
\label{ssec:UVMerit}

In the process of evaluating the UV concept, we believe in the necessity of respecting a few basic conditions, the recognition of which forms the basic premise of this paper, which is also consistent with UN SDG: \enquote{inclusive, safe, resilient, and sustainable} as discussed in section~\ref{ssec:UN-development-goals}.

Firstly, the UV concept is built upon the awareness of responsibility and peril. In fact, the exploration of the UV concept stems from the recognition of collective responsibility for mankind to solve the challenges we face, and the proactive danger-aversion that results from such responsibility. We are convinced that with influence from scientific predictive studies, new feasible solutions, as well as with proactive intervention of our lifestyles, it is possible to alleviate or even prevent the exacerbation of multiple ecological and societal crisis that we currently face, and to make our future system to be more \enquote{safe, resilient and sustainable}. 

Secondly, the UV concept involves an understanding of diversity and inclusivity. A true Universal Village should recognize diversity in individuals, groups, and cultures, as well as encourage such diversity as the source of creativity and enrichment of human lives. Therefore, the designs of the Universal Village and its components should reflect a sense of inclusiveness that discover and make use of the potential positive impact of such inclusive philosophy and enhance the robustness of future society. Such mindset echoes the UN SDG mentioned in section~\ref{ssec:UN-development-goals}.

Thirdly, the UV concept pursues a goal of shareability and sustainability. UV strives to utilize the technological advancement of the Internet age to construct an egalitarian society in which more people can benefit from the progress of technology, and feel the comfort, convenience, and happiness that it brings. The process of sharing is economization of resources, as it makes maximal use of such resources at the same cost.  In the process, we also aim for beneficial transmission, pairing, and recycling of data, broadening the definition of \enquote{sharing} to include not only material redistribution, but also that of capability. In terms of material redistribution, products or parts may be shared between entities. Space is another resource that may be shared and is currently done through Peer-to-peer (P2P) storage and home rental services. Capability sharing manifests itself through areas of cloud manufacturing, carpooling, and medical assistance, etc. Such redistribution and reallocation of resources can help in promoting development of a more sustainable environment.

\subsection{UV Elements}
\label{ssec:UVElements}

As introduced in paper~\cite{mit-uv2013}, the final objective for Universal Village is to achieve human-nature harmony through technology. Universal Village is defined as the multi-functional multi-format sustainable communities incorporating the ideal mixture of city and suburban areas, and use need-based human-nature-oriented systematic top-down design approach~\cite{mit-uv2013}.


The four basic UV elements are:
\begin{itemize}
\item    Development of new energy source
\item    Development of new material
\item    Development of effective microbial technology and environmental protection technologies
\item    UV Lifestyle enabled by Information technology
\end{itemize}

% Basically, these elements work cooperatively in typically intelligent subsystems, including smart home & communities, intelligent transportation, smart healthcare, and smart environmental protection, smart food systems, which are discussed in this paper. 

Basically, these elements work cooperatively in typically intelligent subsystems, including Smart Home \& Community, Smart Medicine \& Healthcare, ITS, Urban Planning \& Crowd Management, Smart Energy Management, Smart City Infrastructure, Smart Response System for City Emergencies, Smart Environmental Protection and Smart Humanity, etc. The listed subsystems are mainly discussed in this paper.



\subsection{UV Design Process}
\label{ssec:UV.Design}

As introduced in paper~\cite{mit-uv2013}, the three steps of UV system design are shown below. 


\begin{enumerate}
	\item Understand Mutual Interaction for Multiple Inputs \& Multiple Outputs Systems (MIMO analysis)
	\item De-couple MIMO Systems based on Dynamics Principal Component Analysis. 
	\item Design UV Feedback Control System 
		\begin{itemize}
			\item Implement \textit{Sensing \& Evaluation}
			\item Implement \textit{Decision-Making}
			\item Implement \textit{Action}
		\end{itemize}
\end{enumerate}

% It is worth mentioning that the feedback loop is the key for successful management of UV systems. 
It is worth mentioning that the feedback loop is the key element to ensure the successful implementation of system control for any systems. 
Specifically, UV feedback loop design are shown in Fig.~\ref{fig:fyj:UV_feedback_loop}. 
For each intelligent subsystem, we should ensure that the system loop is closed. Information/data are first collected or acquired and then sequentially circulated in the loop of data acquisition, cloud storage, data profiling, decision making and optimization, and \enquote{action}. During the procedure of information dissemination, the technologies of mobility, communication and IoT are involved.  A typical feedback system would consist of four elements, including \enquote{decision making}, \enquote{action},  \enquote{data acquisition} and \enquote{communication}, as shown in 
Fig.~\ref{fig:jz:smart_home_feedback},~\ref{fig:cgh:smart_healthcare_feedback},~\ref{fig:gwy:ITS_feedback},~\ref{fig:cyx:smart_energy_feedback},~\ref{fig:zlf:smart_infrastructure_feedback},~\ref{fig:zj:smart_emergency_feedback},~\ref{fig:zzj:smart_environment_feedback},~\ref{fig:css:smart_humanity_feedback}, which will be discussed in detail later.

\begin{figure}[h!]
        \centering
        \includegraphics[width=0.7\linewidth]{images/fyj-UV_feedback_loop}
        \caption{Feedback Design Scheme for Universal Village.}
        \label{fig:fyj:UV_feedback_loop}
\end{figure}


Besides the MIMO analysis, system dynamics is also an effective tool for analyzing the complicated systems. 



\subsection{UV Connectivity at Different Levels}
\label{ssec:UV.Connectivity}

Under the UV framework, we explore how human, system, nature are connected at different perspectives of lifestyles as below. The first two items are to describe general connections for human being, in both informational level and material cycle. Specially \textit{Material Cycle} is an important element for UV Lifestyles which explore how we interact with our mother nature, how we consume natural resources and process the waste, which is very important in order to increase the human’s awareness for the environmental challenges. 


\begin{itemize}
\item Information Dissemination as in section~\ref{sec:Information_Dissemination}
\item    Material Cycle as in section~\ref{sec:UVMaterialCycle}
\end{itemize}


Then we discuss the connections for the following eight UV elements describing different activities for human beings. UV framework follows system law and ensures these elements work cooperatively. For example, in the context of UV lifestyle, Intelligent Transportation Systems are not only about commuting and mobility, but are closely related to Crowd Management, Urban Planning, Healthcare and Well-being, etc. Smart Healthcare involves not only about medicine and treatment but also general well-being, road-rage control through ITS, etc. % ease road-rage through ITS, etc.


\begin{itemize}
                \item    Smart Home and Community
                \item    Smart Medicine and Healthcare
                \item    ITS, Urban Planning and Crowd Management
                \item    Smart Energy Management
                \item    Smart City Infrastructure
                \item    Smart Response System for City Emergency
                \item    Smart Environmental Protection
                \item    Smart Humanity
\end{itemize}

Specifically, there are three categories of connections. 
\begin{itemize}
	
	\item   Explore the mutual interactions among different intelligent systems or element. 
	
	Section~\ref{sec:InteractionSubsystems} summarizes the comprehensive mutual relationship among eight key UV subsystems. Additionally, interaction of different systems may be conducted on an event-centric basis, forming a self-propagating event chain, so that new changes can successfully influence the entire network.
	            
    \item    Ensure that feedback control loops are closed. 
	
	Section~\ref{sec:Subsystems} summarizes the detail components of  \enquote{data acquisition}, \enquote{decision making}, \enquote{action}, for eight key UV subsystems as shown in 
Fig.~\ref{fig:jz:smart_home_feedback},~\ref{fig:cgh:smart_healthcare_feedback},~\ref{fig:gwy:ITS_feedback},~\ref{fig:cyx:smart_energy_feedback},~\ref{fig:zlf:smart_infrastructure_feedback},~\ref{fig:zj:smart_emergency_feedback},~\ref{fig:zzj:smart_environment_feedback},~\ref{fig:css:smart_humanity_feedback}.
	
	\item    \textit{Situ-remote hybrid} for control signals. 
	
	Control signals can be sent to targets in three possible manners. 1. In Situ connection  2. In Situ-Remote hybrid  connection 3. Remote connection.
	
	Uber mode is an example of  \textit{in situ-remote hybrid} where connection between control and targets are dynamics. They can maintain the relationship of virtual connection remotely or actually get together and meet. 
	
	Subsection~\ref{ssec:cgh:smart_health} proposes two modes of \textit{in in situ-remote hybrid}  relationship between patients and doctors. A intelligent system \textit{Doctor Uber}  is proposed to send doctors to patients when disasters or urgent situations happen~\cite{uv2018-healthcare-ma.haoran}. A intelligent system  \textit{Medical-trip} is proposed to bring patients to doctors in foreign countries and to construct evaluation index for medical tourism centers based on which patients can choose suitable medical destinations in foreign countries~\cite{uv2018-healthcare-su.qikai}.

\end{itemize}

% \subsection{System Dynamics}

\section{Information Dissemination: Mobility, Connectivity and IoT}
\label{sec:Information_Dissemination}

While the subsection~\ref{ssec:UV.Connectivity} mainly discuss the ideal connectivity and information flow, we will discuss and evaluate the supporting technologies in this section. 

\subsection{IoT and Smart Monitoring}
\label{ssec:IoTmonitoring}

As shown in Fig.~\ref{fig:fyj:iot_market}, demand for data, and therefore sensors, has soared since the start of the Industrial Internet of Things (IoT) revolution. Now, the types of sensors have started to vary further to include acoustic, ultrasonic, infrared, video, and industrial machine vision equipment.
And IoT market would be expanded quickly, according to the research in Fig.~\ref{fig:fyj:iot_market}, the total IoT platform revenue in 2026 will increase to 60 billion dollars which is more than three times higher than that in 2012.


With such infrastructure and platform, we can monitor and collect several important measurements at multiple sensors, transmit them to the data terminal, equipment Bluetooth and computing center, and make them available at cloud and Internet. Theoretically minor changes should be able to be identified and enable real-time response. The accumulation of historical data can help with prediction model and more in-depth data-mining, thus generating data reports for decision makers.
% Small changes in data are obtained from multiple sensors and transmitted to the data terminal, equipment Bluetooth and other forms, which are transmitted to the Internet through routing. After that, it is uploaded to our computing center to draw the final conclusions through mathematical models and generate data reports for decision makers to make decisions.


\subsection{Smart Agent}
\label{ssec:smartagent}

Smart agents are important components in the information flow to ensure connectivity between  human and nature. % Voice-based interface ensures that users can still have access to intelligent systems while human operators are not available.
Section~\ref{ssec:zcy:smart_home} has thoroughly introduced several typical smart agents used in smart home which interact with human being conveniently. On the other hand, the limitation discussed in section~\ref {ssec:AI.limitation} does apply to these smart agents and should raise people’s awareness. Besides, people are are also concerned that the smart agents might have negative impact on the etiquette of young generation. It was reported that Alexa generation may be learning bad manners from talking to digital assistants~\cite{alexa-generation}.
%https://www.telegraph.co.uk/news/2018/01/31/alexa-generation-could-learning-bad-manners-talking-digital/

\subsection{Intelligent Systems} 
\label{ssec:jz:intelligent_systems}


Many current intelligent systems employ the old top-down hierarchical design (cf. \ref{ssec:zcy:smart_home}), but the trend in the field of intelligent system development is to move toward a more decentralized design, especially in the domain of workload management~\cite{trudnowski2006power, , guerrero2013advanced}. 

Fabien Sator from \textit{Université de Grenoble} proposes an  Input/Output Symbolic Transition Systems (IOSTS) in~\cite{sartor:tel-00748676} which stresses on the mathematical modeling of a given reactive system as a whole. IOSTS views each automaton as a directed graph where the vertices are states and configurations and the arcs are transitions. By this abstraction, Sartor links every possible command to an automaton to an arc in the graph. The main focus of the thesis is on accurate description of the states and state changes so that two distinct systems can be linked together to form a composite system if needed~\cite[p.~56]{sartor:tel-00748676}.


\subsubsection*{Challenges in Adaptability} 
\label{sssec:jz:intsys:adapt}


Recent developments in intelligent systems have shown effort in adapting to their role on a case-by-case basis, however, many systems are designed for their particular locale and lacks the ability to take into consideration peculiar cases of their users. Examples of such peculiar cases include the naming of Icelandic users~\cite{scott2014names} and various other localization problems~\cite{computerphile2014internationalizing,computerphile2014problem}. In the theme of information dissemination, it can be categorized as overreliance on detailed and context-specific models. Referring to the example of Icelandic naming, the failure of handling Icelandic names usually comes from a lack of understanding in how Icelandic names are formed. Without proper knowledge there cannot be any accurate modeling or ruleset regarding the names, and therefore systems lacking in those aspects fail to deal with Icelandic names.


\section{UV Material Cycle}
\label{sec:UVMaterialCycle}

After investigating the information flow among the connected systems, we further explore the connectivity in physical form,  namely the  material cycle. 

In section~\ref{ssec:urbanization-challenges}, we present the high demand of natural resource due to urbanization, which further causes large amount of waste as shown  shown  in Fig.~\ref{fig:fyj:gdp_resource} and  Fig.~\ref{fig:fyj:waste_country}. As shown in the Fig.~\ref{fig:fyj:waste_technology}~\cite{UN-world-energy-report}, majority of waste is sent to landfill. There are huge challenges on the way of improving living standards under the premise of no under the premise of no environmental degradation.


\begin{figure}[h!]
        \centering
        \includegraphics[width=0.7\linewidth]{images/fyj-waste_by_technology}
        \caption{Amount of waste disposed by technique (2012)}
        \label{fig:fyj:waste_technology}
\end{figure}


We would like to walk through the journey from natural resource, to product, and to waste, i.e., the whole life cycle of products, and to explore how to reduce the demand and waste during the journey. The involved factors include lifestyle, customer needs, product design, product manufacturing, feedback collection from customers, reuse and recycle, scraping and waste processing. 


\subsection{UV Lifestyle, Custom Needs, and Design}
\label{ssec:UVLifesyle}


With the advancement of technology and science, human society has been building more and more factories to produce a wide range of products to satisfy people’s need. Some of the products may be unnecessary in people’s life.
As mentioned in section~\ref{ssec:urbanization-challenges}, current lifestyle consumes huge amounts of materials rapidly, so it is not sustainable in the long term. Consequently, significant amount of waste materials are produced, which leads to large amount of waste-disposal that put serious pressure on natural environment.  Thus, people need to explore how to innovate our current lifestyle and to seek a feasible balance between environmental protection and improvement of living standards. To achieve such an objective, we should should first explore four essential elements of our daily needs and lifestyle, including, food/water, clothing, housing, and transportation.

%As for analyzing the needs and consumption of life, there are four parts should never be ignored. Food and water, as people’s basic needs, are definitely significant, and so are clothing, housing, and transportation.



%In the first place, food loss makes negative impact in the world. Almost one-third of food would be discarded in the processing of transferring and selling~\cite{FAO2018Food}.

Not many people realize that food loss is a grave issue in the world. Around 1.3 billion pound of food, almost one-third of total food production, is being discarded during the process of transferring and selling~\cite{FAO2018Food}.  Consumers in the developed countries prefer to purchase vegetables and fruits with perfect appearance, so merchants have to discard the rest of the food which could still be edible. Meanwhile, some developing countries may not have enough food to feed their citizens.
In addition, water scarcity is severe in many areas. Limited number of people have access to abundant water within ten minutes’ walk. However, many people have to walk at least half an hour to obtain safe drinking-water. At present, there are still about 885 millions people who who live in areas without access to safe drinking water~\cite{Water2017Water}.


%according to the research, people find a kind of clothing style called \enquote{Capsule Wardrobe}, which recommends that people need 90-100 pieces of clothes each year~\cite{Courtney2011Wardrobe}.
With regards to clothing, it is estimated that one person only needs around 90-100 pieces per year~\cite{Courtney2011Wardrobe} if he/she follows clothing style of \enquote{Capsule Wardrobe}. 
In contrast, people always purchase new clothes for fashion purposes and rarely wear them. In US, 13 million tons of clothes are trashed per year, which is around 85 percent of all retailed clothes~\cite{cloth-waste}.
%This behavior is definitely a waste of materials and makes negative impact to the environment. 
Recent investigation indicates that producing a T-shirt or a pair of jeans costs about 5,000 gallons~\cite{Glynis2015Dirtiest}, which is equal to the monthly usage of water for an American, or the annual consumption of water for a person who lives in Sub-Saharan Africa.

%In the third place, housing consumption also plays an essential role in the lifestyle, especially on electricity consumption. In the electrical era, people could hardly live without electricity. For instance, people would use the air-conditioning and heating system during home life. A house consumes about 3,826 watt-hr electricity per day, which is such a huge waste~\cite{gifford2012residential}.

Regarding housing, the average electricity consumption has been rapidly increasing due to urbanization and high consumption of electricity on air-conditioning, heating systems and so on. The electricity consumption for each family varied significantly among different countries, from 74 kwh in Nigeria, 731 kwh world average, to 4517 kwh in USA and 4741 kwh in Canada per person per year~\cite{avg-household-electricity}~\cite{gifford2012residential}. %  while daily usage on air-conditioning and heating system is huge. A big house consumes about 3,826 watt-hr electricity per day, which is such a huge waste~\cite{gifford2012residential}.
 

As for dangers in our attitude on transportation, which impacts people's life deeply, there are three severe problems that should not be neglected: traffic congestion, parking system, and car accidents, all of which influence people's physical and mental health. For example, the Road Rage is a type of result under the negative emotion. In US, the financial cost on traffic jams and searching parking spots is respectively around \$305B and \$73B per year~\cite{traffic-cost2017}~\cite{parking-cost}. Therefore, scientists should seek new methods to deal with those problems.


% As shown in A report from the AAA Foundation for Traffic Safety states that at least 218 people have been murdered and 12,610 people have been injured in 10,037 road rage incidents it examined ~\cite{Olivea2017Effects}. 




The above discussion shows that individual’s lifestyle plays important roles regarding to the consumption of food, water, cloth, products, electricity, gas, etc. With collective effort, we can choose to increase or decrease natural resource-consumption and waste-disposal. % can be expanded.
Of course, we should not ask people to give up their comfort, convenience and enrichment of life for environmental benefits. The objective of Universal Village is to explore the feasible technologies and to reach a balance. 
For example, if the feedback of customers can be sent back to manufacturers before the production process,  design problems can be identified at the early phase which would help to reduce waste-disposal. Instead of producing many low-quality products which would be discarded by customers quickly, we can produce high-quality products to reduce resource waste. To reduce the high cost associated with exquisite products and make them affordable to their customers, sharing business model and innovative manufacturing industry would both help. 
If the re-usable components of broken or waste products can be reused or recycled at early phases before the waste products are fully scrapped, we may significantly reduce the demand of natural resources and man-made resources
%manufacturing task for relevant components from raw materials directly.
Details are discussed below.

\subsection{Intelligent Manufacturing in Universal Village}
\label{ssec:intel_anufacturing}

Intelligent manufacturing is an important aspect of Universal Village. 
To achieve better sharing and sustainability of UV system, material recycling is a very important element. Intelligent manufacturing based on IT and AI provides powerful support and foundation for efficient material circulation, thus enabling the effective transfer, matching and utilization of materials in UV, and promoting the sustainable development of human civilization.

\subsubsection{New Challenges in Intelligent Manufacturing}


Recently, the development of new technologies, like cloud technology, AI, and IoT, has affected manufacturing industry deeply. But there are still some issues need to be solved. For example, there is an imbalance between small and medium-size enterprises and large-size enterprises in manufacturing requirements and resources. Small and medium-size enterprises can produce what customers need but they lack resources, while large-size enterprises has resources but the equipment utilization is usually low. The global manufacturing mode is in transformation from networked manufacturing to intelligent manufacturing which evokes new challenges such as system robustness, task scheduling, multi-tenant data management, information security and so on. 

\subsubsection{Cloud Manufacturing Platform}

One solution to the challenges mentioned above is to build a cloud manufacturing platform. 
Cloud manufacturing is a networked, service-oriented model, and its important characteristics include collaboration, personalization, and intelligence. The aim of cloud manufacturing platform is to break down the information barriers, to integrate the manufacturing resources and capabilities of different providers, and to provide improved on-demand manufacturing services to customers through IT/OT network, knowledge and data management, and logistics services~\cite{zhou2018modelling}.There are three stakeholders in the cloud manufacturing platform, and they are including demanders, providers and operators. The concept of cloud manufacturing was proposed in China~\cite{li2010cloud}, and the idea of Smart Cloud Manufacturing has been explored.  At present, some new progress on cloud manufacturing has been made in standards, technologies, platforms and industrial applications. 

\subsubsection{Key Technologies and Application}

With the help of emerging technologies such as manufacturing, information and communication, and artificial intelligence, a user-oriented, unified manufacturing cloud platform is built to provide customers with on-demand manufacturing resources, services and capabilities anywhere and anytime through terminal devices and cloud platform. There are multiple key technologies in cloud manufacturing such as data perception and acquisition~\cite{tao2014iot}, individualized requirements~\cite{zhou2018diverse}, cloud service composition~\cite{zhang2010flexible} dynamic service scheduling~\cite{zhou2018event}, cloud 3D printing services~\cite{zhou2018multi}, etc. In cloud manufacturing, service scheduling is a very important aspect in optimizing the task execution process. As a special type of manufacturing services in cloud manufacturing, logistics services provide transportation to guarantee just-in-time delivery of products from distributed providers to demanders. In China, cloud manufacturing has been applied in some industries such as automobile, aerospace and numerical control machining. The applied technologies include cloud simulation, multi-tenant management, 3D printing, big data, etc. 
% The application in one of the Chinese companies includes cloud platform, collaborative R\&D, manufacturing big data and knowledge services.

\subsection{Dissembling, Recycling and Waste Processing}

Metal scrap recycling industry~\cite{uv2018-smartenvironment-tan.tian} is now widely integrated in socioeconomic system including secondary metal commodity trading, residential and industry waste recovery and reuse, and environmental protection. While metal scrap recycling delivers the most basic needs to ensure social functioning, it comes at a high cost due to its primitive mode of operation. 
Scrap recycling industry has 4 key factors, identifying materials, matching scrap with user, processing materials, and controlling pollution. 
Scarce labor resources and scrap knowledge become main constraints in all 4 factors of the scrap recycling industry. Accordingly, technologies such as AI and big data are critical to the all-round upgrade of metal scrap recycling in modern society, which may increase efficiency of all factors to dramatically improve the recovery ratio of renewable scrap sources without sacrificing natural environment or increasing social operation cost. 

A comprehensive metal scrap recycling system should be designed based on AI and big data toward the goal of enabling efficient recycling of metal scrap. Specifically, it will provide an overview of current metal scrap recycling industry, highlight key technologies, such as computer vision and MPIRAN (Max-error Pruning Improved RAN) algorithm, could be used in identifying, matching and processing. Various system design methods can be applied to reform metal scrap recycling industry chain. As more researchers and practitioners realize the seriousness and get started in this field, the important benchmarking metrics and design considerations will be used for evaluating the complex metal scrap recycling system in terms of design and operation process.

Besides, the waste management system attracts more and more attention around the world due to the rapid increase in waste accumulation in recent decades along with economic development. We need to analyze the current waste management systems and the correlation between the effectiveness of a country’s waste management systems and the country’s developmental conditions. It is worth noticing how new technologies innovate the current waste management systems. What’s more, the guideline on how different countries should innovate their current waste management systems should be proposed and we need to work together to solve the current environmental crisis~\cite{uv2018-wastemanagement-wen.shunzhi}.


In short, we  explore how to reduce the demand and waste through studying the journey from natural resource, to product, and to waste, and innovating the lifestyle, customer needs, product design, product manufacturing, feedback collection from customers, reuse and recycle, scraping and waste processing. 


\section{Subsystems}
\label{sec:Subsystems}

Here below we will start to present eight typical subsystems that are vital to smart cities.

\subsection{Subsystem: Smart Home and Community} 
\label{ssec:zcy:smart_home} %{ssec:jz:smart_home}

As people spend a significant portion of their time at home, and as homes form an important aspect in people’s lives, intelligent improvement to smart home technologies can ameliorate the quality of life in general. Seeing that humans, as a gregarious species, also spend much time socializing, the extension of smart home into the community is also an important component of our Universal Village ideal.

\begin{table*}[t]
\small
\caption{Typical Events and Their Connections}
\centering
\begin{tabular}{|P{2.5cm}|P{2cm}|P{2.5cm}|P{3cm}|P{2.5cm}|P{2.5cm}|}
	\hline
	       Event         &      Category      &      Triggered by      &         Potential result          &       Priority       &                  action                   \\ \hline\hline
	        rain         &      weather       &   weather broadcast    &             flooding              & (depend on severity) &     user notification; close windows      \\ \hline
	      flooding       & weather; emergency &         sensor         &          property damage          &    0 (emergency)     &       user warning; emergency call        \\ \hline
	    unlock fail      &      security      &       smart lock       & intrusion attempt; unlock success &   4 (unimportant)    &             user notification             \\ \hline
	repeated unlock fail &      security      &       smart lock       & intrusion attempt; unlock success &     3 (neutral)      &        user notification; timeout         \\ \hline
	  network failure    &      network       & interrupted connection &                                   &    2 (important)     & diagnostics; attempt of alternative route \\ \hline
	    house empty      &      activity      &    presence sensor     &                                   &     5 (trivial)      &    cut power to unnecessary appliances    \\ \hline
	  authorized entry   &      security      &       smart lock       &                                   &   4 (unimportant)    &     notification; doorway monitor on      \\ \hline
	 reported incident   &     (generic)      &  (inter-system comm.)  &                                   &         N/A          & notification; preliminary categorization  \\ \hline
\end{tabular} 
\label{table:james}
\end{table*}

\subsubsection{Current Challenges}
\label{sssec:zcy:smart_home:challenges}

Despite the recent progresses made in the smart home system industry, the current commercial solutions share many limitations. In~\cite{petrov2018home}, Petrov describes a basic structure of a smart home system, which is similar to the model used by most prevalent commercially-available smart-home systems. The centralized design is prone to fatal failure upon malfunction of the central node. If the central component of the smart home system malfunctions, the user will not be able to interact with the appliance through the system, effectively rendering the system useless. In addition, this paradigm of operation is reactive instead of proactive see Fig.~\ref{fig:jz:smart_home_lv_1}. User commands are required to trigger the appliances, and therefore the system is incapable of adapting to situations that the user has not foreseen. Among other things, it should be possible for devices and appliances to directly communicate among themselves, with the central unit taking on an adjudicator role for conflict resolution. 

\begin{figure}[h!]
        \centering
        \includegraphics[width=\linewidth]{images/jz_smart_home_lv_1}
        \caption{Smart home system with \enquote{level 1} intelligence: the central unit (CU) controls the appliances and relays user command. The arrows represent the unidirectional interaction between the CU and the appliances; the dashed line represents the interaction (command and feedback) between the system (represented by the CU) and the user.}
        \label{fig:jz:smart_home_lv_1}
\end{figure}

Smarter home automation systems exist. Some commercial solutions, such as Nest Thermostats are capable of learning from user routine, and can preemptively adjust itself prior to user action~\cite{pogue2011thermostat}. However, the preemptive behaviour is limited to the appliance due to the lack of a common communication protocol, as shown in Fig.~\ref{fig:jz:smart_home_lv_2}.

\begin{figure}[h!]
        \centering
        \includegraphics[width=\linewidth]{images/jz_smart_home_lv_2}
        \caption{Smart home system with self-learning and self-regulating appliances (\enquote{level 2} intelligence): the appliance can make predictions and adjust itself, but such self-learning does not have an impact on the system as a whole. Similar to that in a \enquote{level 1} smart home system, the interaction between the user and the system is done through the CU.}
        \label{fig:jz:smart_home_lv_2}
\end{figure}

\subsubsection{Smart Home Framework and Functionality}
\label{sssec:zcy:smart_home:framework}


In view of a Universal Village concept, smart home systems that fit into the universal subsystem framework should include typical feedback-control loops while the concrete strategies and techniques for all elements as illustrated in Fig.~\ref{fig:jz:smart_home_feedback} in order to provide an adept and  decentralized solutions.

%In view of a Universal Village concept, smart home systems should be concluded and fit into the universal subsystem framework. In reference to the paradigm in Fig.~\ref{fig:jz:smart_home_feedback}, in order to provide adept and  decentralized solutions, smart home systems implement feedback-control framework with concrete strategies and techniques.


\begin{figure}[h!]
        \centering
        \includegraphics[width=\linewidth]{images/fb_home}
        \caption{Ideal Feedback Control for Smart Home System}
        \label{fig:jz:smart_home_feedback}
\end{figure}

To counteract the problem inherent in hierarchical designs, a decentralized system can be designed. Compared to a hierarchical system, a decentralized smart home system has the following advantages:

\begin{itemize}
        \item Failsafe: if a component malfunctions, it can effectively be isolated and excluded from the system while the rest of the system coordinates to compensate the loss in processing power;
        \item Self-regulating: any change in one component can trigger response potentially from the entire system;
        \item Modular: addition and removal of components can be done spontaneously without needing to shutdown the system (hotfixes)
\end{itemize}




\subsubsection{Introduction and Evaluation of Current Technologies}
\label{sssec:zcy:smart_home:evaluation}

\paragraph{Alexa by Amazon}

Amazon Alexa, the virtual assistant developed by Amazon, offer third-party integration frameworks through their respective APIs, allowing for creation and development of third party extensions to the systems. Amazon in particular has integrated the process of publishing and acquiring the extensions into their website~\cite{amazon2018understand}. In October 2017, Amazon reported a figure of over 25,000 Skills being developed and published by third-party developers~\cite{businesswire2017amazonQ3}. However, the Alexa-centric system is restrictive in its scope; while the smart speaker itself is able to access the Internet, it is unable to apply its Internet connection onto the smart home system. As far as the smart home system is concerned, no outside connection is available yet.

\paragraph{Google Home by Google}

Google Home is a product line of smart speakers by Google featuring the company's personal assistant, the Google Assistant. Similar to its counterpart produced by Amazon, Google Home allows for user interaction with and control of pre-paired smart home devices through voice~\cite{bohn2016googlehome}. Google Home is well integrated into other Google services. Nonetheless, the reliance on user command and thus lack of predictive behaviour is evident on Google Home. Google Home is generally less restrictive in scope than Amazon Alexa, due to its ability to perform Google searches, but Google Home keeps the search engine functionality separate from its smart home system component.

\paragraph{HomeKit by Apple}

Apple's HomeKit framework offers to the user the ability to define room structures and access rights. By defining rooms, the user can more precisely refer to a subgroup of appliances when communicating with Siri, the iOS smart assistant~\cite{ritchie2014homekit, wollerton2015control}. In addition, the usage of the Home app on an iOS device allows for control outside the domestic wireless network. This makes HomeKit more decentralized than most alternatives, which more often than not revolve around a central hub, usually the smart-speaker~\cite{crist2018what}. 
Siri also incorporates voice fingerprint recognition that aims to match voice profiles to users, in order to prevent impersonation and access to personal data~\cite{siri2018personalized}. However, Siri does not have the best speech-recognition quality out of the three commercial products~\cite{wollerton2018siri}, and the HomePod variant is more limited in functionality than the regular version installed on iOS and macOS. In~\cite{siri2018finding}, Apple's Siri Speech Recognition Team describes a technology presumably absent on HomePod but present on iOS devices.


\subsubsection{Interaction with Other Systems}
\label{sssec:zcy:smart_home:interaction}

The IOSTS in~\cite{sartor:tel-00748676} discussed above in section~\ref{ssec:jz:intelligent_systems} describes a way to form composite systems that can interact with each other while still keeping private data within the respective systems. Although the IOSTS does not propose a technical standard, the mathematical modeling is strongly applicable to the field of smart home systems. To ensure the physical and psychological wellbeing of the occupants in a home, it is necessary to collaborate with intelligent systems of medical care and emergencies. As the telephone system, spanning an entire city, can connect smart home systems and emergency dispatch centres, automated communication between the two is possible.

\paragraph{Smart Energy}




Furthermore, as a system requiring close to 100\% uptime, a smart home system needs to be energy-efficient, hence depends on and contributes to the local smart energy system. Through smart monitoring of appliance energy usage, smart home systems can collect data useful for making decisions on energy allocation.

\paragraph{Smart Infrastructure}

A smart community is not solely comprised of individual smart home systems, but rather also includes communication with city infrastructures. Any change in the city infrastructure (road construction, etc.) can have a profound impact on smart home systems, and homes can also send data (with the users' consent) to aid in accurately pinpoint any issue that require response from the community.

\subsubsection{Discussion and Summary}
\label{sssec:zcy:smart_home:discussion}

One aspect worth noting is the interaction between the smart home system and its users. Distinguishing smart home systems from other intelligent systems is the fact that, unlike other specialized systems, the user of a smart home system could be anyone. It is therefore crucial that the system designs take into consideration the different situations of its users and adapt to it, as discussed in \ref{sssec:jz:intsys:adapt}, as well as the requirement on \enquote{inclusive, safe, resilient, and sustainable}, the development goals set by UN SDG.

%>>>>>>> JZ - 2018-12-26


\subsection{Subsystem: Smart Medicine and Healthcare}
%\label{ssec:SubsystemHealthcare}
\label{ssec:cgh:smart_health}

In order to meet the basic need of mankind, healthcare services must combine effectiveness as well as safety. Therefore, computer engaged techniques, as smart healthcare methods, should consider practicality as well as credibility, which must be guided by a UV paradigm. 


\subsubsection{Current Challenges}
\label{sssec:cgh:smart_health:challenges}

For a long time, empirical researchers have been seeking a way to integrate computer power with the existing healthcare system. In applications related to medical cares, industries and hospitals have called for a solution making use of the development in computer science and electronic engineering disciplines. Integrated systems where patients, physicians, sensors and computers get connected to personalized healthcare solution, hence come into being. In adapting computational and electronic technologies into healthcare applications, achievements have been made but challenges and drawbacks persist.


\paragraph{Growing expenditure}


In general, a smart healthcare system is sophisticated and hence expensive. The expense brought by personalized healthcare system does not always improve the medical quality. Starting from the 21st century, the rate of health expenditure to a gross domestic product has been increasing continuously. While people around the world are getting wealthier, expenditure in healthcare keeps growing~\cite{hc_paper_34}. Expense and effect need to be considered combined with a reasonable heuristic function.


\begin{figure}[h!]
        \centering
        \includegraphics[width=\linewidth]{images/cgh_health_care_expenditure}
        \caption{Health Expenditure is in Gross Domestic Product}
        \label{fig:cgh:expenditure}
\end{figure}

\paragraph{Healthcare network}


In a smart healthcare system, healthcare network bridges computer, mobile, wearable sensor, wireless sensor and communication environments. It has innovated the original way of communication between medical workers and patients by enhancing the computing and analysis capabilities of portable devices such as computers and mobile phones. The current healthcare network allows patients to receive timely feedback on their physical condition and corresponding safety warnings in any environment. Challenges in developing healthcare network are related to coordinating with advanced information collecting endpoints, performing information processing and information feedback~\cite{hc_paper_14},~\cite{hc_paper_15}.


\paragraph{Machine learning in healthcare}


In smart healthcare systems, machine learning technologies are used to extract descriptive information from structured features and perform medical decision making. Tracing back to late 1900s, a so-called clinical decision support system has been proposed and aimed to improve diagnosis quality of healthcare service~\cite{hc_review_3},~\cite{hc_review_5},~\cite{hc_review_3}. These attempts achieved some success in disease preventions and daily care but failed in helping increase diagnosis quality mainly because of limited computation power.



By increasing hardware support, machine learning models have achieved human-level performance in many healthcare fields~\cite{hc_paper_6},~\cite{hc_paper_10}. With well-developed machine learning models~\cite{hc_paper_24},~\cite{hc_paper_25},~\cite{hc_paper_26}, empirical researchers, however, have identified that common symptoms in decision making can result in severe problems in application related to medical and health. According to~\cite{hc_paper_7,hc_paper_10,hc_book_9}, a false decision, especially false positive classification made by model and malicious data generated by attackers, can prompt distress and distrust in decision systems.


Also, to provide healthcare service with machine intelligence calls for more human-centered considerations, especially in interpretability, reproducibility, and generalizability. Models with high accuracy but low interpretability might not be eligible to applications, given concerns in a requirement for rigors. In this sense, statistical modeling like deep neural networks~\cite{hc_paper_24},~\cite{hc_book_9} need to be further examined.


\paragraph{Big data in healthcare}


In order to ensure healthcare decision making applicable, requirements have been put upon data resources for machine modeling. The base data for a training process should be unbiased and same to the application population. For the growing demands in healthcare applications, attempts have been made to construct medical datasets with structured and trustable data~\cite{hc_dataset_uci}. To ensure the quality of modeling, the quality of ground truth must be granted. Secondly, in order to certify enough information for a decision making~\cite{hc_paper_27}, features included in dataset must be coordinated and related to the modeling target. In a scenario of healthcare, these concerns combined request for more expertise participating.


The difficulty of big data application in healthcare lies not only in the speed and amount of data generated but also in the various sources where data comes from~\cite{hc_paper_28}. Many network architectures for the healthcare system have been proposed and examined~\cite{hc_patent_11} -~\cite{hc_paper_14}, where an important function is to provide data sources for an analysis purpose. Apart from human expertise or knowledge-based decision system, current machine learning modeling can achieve a better performance only in a case when a large amount of data is available; in this sense, unifying exchanging format and features will for sure increase the available data source, hence promoting modeling performance. Existing healthcare systems are facing troubles exchanging medical records universally and bringing out a unifying format, which will certainly influence their functionalities.

Privacy is another concern regarding big data applications in healthcare. Healthcare user groups are vulnerable to malicious attacks making use of leaked medical information~\cite{hc_paper_8},~\cite{hc_paper_28},~\cite{hc_review_29}. The scope of healthcare big data applications and privacy protection must be carefully examined.


\paragraph{Medical trial norms}



Before putting into operation,  % Before putting onto use, 
a smart healthcare technique must undergo a careful examination first. The traditional medical experiment follows follows the strict flow. 
The traditional trial paradigms, however, are not suitable for computational techniques used by the integrated healthcare system. Given fact that medical computational techniques often follow a computer science design and does not impact treatment directly, a universally acceptable and reasonable experiment flow needs to be designed.


\subsubsection{Smart Healthcare Framework and Functionality}
%\label{ssec:feedbackloop_smart_healthcare}
\label{sssec:cgh:smart_health:framework}

% In view of a Universal Village concept, smart healthcare systems should be concluded and fit into the universal subsystem framework. In reference to the paradigm in Fig.~\ref{fig:cgh:smart_healthcare_feedback}, in order to provide personalized healthcare solution, integrated medical systems implement feedback-control framework with concrete strategies and techniques.

In view of a Universal Village concept, in order to provide personalized healthcare solution, smart healthcare systems that fit into the universal subsystem framework should include typical feedback-control loops while the concrete strategies and techniques for all elements as illustrated in Fig.~\ref{fig:cgh:smart_healthcare_feedback}. 

\begin{figure}[h!]
        \centering
        \includegraphics[width=\linewidth]{images/fb_healthcare}
        \caption{Ideal Feedback Control for Smart Healthcare System}
        \label{fig:cgh:smart_healthcare_feedback}
\end{figure}


\subsubsection{Introduction and Evaluation of Current Technologies}
\label{sssec:cgh:smart_health:evaluation}

In this section, we will respectively present the technologies relevant to four elements, \enquote{Data acquisition \& Communication}, \enquote{Decision making}, and  \enquote{Action},  shown in Fig.~\ref{fig:cgh:smart_healthcare_feedback} and then examine them in both their own functional limitations and a Universal Village point of view. 

From the perspective of time and space, technologies in a healthcare system can be classified respectively. By doing so, researchers identify techniques with their own functionality and scenarios. In a spatial dimension, researchers divide smart healthcare methods into in situ and remote; in a time dimension, medical solutions can be classified regarding time relationship to diagnosis.

\paragraph{Data acquisition \& communication}

\begin{itemize} 

\item Healthcare monitoring enabled by the Internet of Things

Surveillance applications in healthcare systems depend on sophisticated sensors. Types of a medical system sensors vary from multiple categories, CCTV, wearable devices, implanted sensors, artificial organs, and cell phones~\cite{hc_survey_12},~\cite{hc_paper_15},~\cite{hc_review_21}.

Medical monitoring endpoints can be made extremely sophisticated. Glucose trackers miniaturized into small chips that can be planted underneath human skins have been put onto the market, freeing patients from stringing measurement~\cite{hc_paper_33}. These instruments extract medical data from users and send information to be collected by corresponding platform or applications. Medical information drawn from the internet of things (IoT) components are meant to be processed by big data functionalities, hence needs related to big data applications in healthcare system should be addressed; that is, unifying data features and measurement must be considered.


Healthcare monitoring techniques are applied in situ, in all time phases, responsible for data acquisition.

\item Electronic medical records


Besides using hard copies for medical information, industries and hospitals have adopted electronic medical record for tracing on different phases of treatment. Despite the achievement to widen exchanging scope, long-term efforts still need to be conveyed towards a unifying record format~\cite{hc_survey_12},~\cite{hc_paper_14}.

The techniques of Electronic-medical records are applied in an \textit{in situ-remote hybrid} way, in all time phases, responsible for data acquisition.
%Electronic-medical records techniques are applied in an \textit{in situ-remote hybrid} way, in all time phases, responsible for data acquisition.

\item Medical networking

With sensors and smart instruments, medical networking techniques aim to construct a healthcare network system to integrate components. Main functionality of medical networking is to deliver healthcare information; given this need, previous researches inspected different architectures of information system~\cite{hc_paper_13},~\cite{hc_paper_14} and homogeneity of information features~\cite{hc_survey_12}. 
With the emerging of new techniques and devices, network design should change accordingly in order to provide precise information and include sophisticated devices.
%With the emergence of new techniques and devices, network design changes to provide more precise information and include more sophisticated devices.



In the sensor network~\cite{hc_paper_15}, now we have wearable sensor system and wireless sensor network, which are essential tools for collecting accurate and reliable information about people's physical condition~\cite{hc_paper_1},~\cite{hc_paper_17},~\cite{hc_paper_14}.  As for the mobile computing network~\cite{hc_paper_15}, smartphones are becoming an increasingly important part in processing and transmitting collected data on people's physical condition~\cite{hc_review_20}. In the back-end network~\cite{hc_paper_15}, the corresponding IT architecture is being designed to meet relevant needs~\cite{hc_survey_12}.


Medical networking techniques are applied in situ, in all time phases, responsible for data communication.

\end{itemize}


\paragraph{Decision making}

\begin{itemize} 

\item Medical image classification


Convolution based artificial neural network architectures have been proved to have human-level performance given enough data. Empirical researchers build machine learning classification models on healthcare dataset and achieved performance close to human expertise~\cite{hc_paper_6},~\cite{hc_paper_8},~\cite{hc_paper_10}. Conveyed by Google, a deep learning model trained on retinal fundus photographs achieved performance better than human expertise in classifying potential diabetic patients~\cite{hc_paper_10}.


The technical challenges for medical information classification lie on its lack of interpretability, mainly because of a black-box nature rooted from machine learning models adopted~\cite{hc_paper_24}, which will for sure harm its trustworthy. On the other hand, failure to deliver large-amount and high-quality training sources also hinders applications of a deep convolutional neural network.

The techniques for medical image classification are applied to diagnosis, responsible for decision making.
% Medical image classification techniques are applied to diagnosis, responsible for decision making.

\end{itemize}

\paragraph{Action}
\begin{itemize} 
\item Surgery robots


Healthcare robotic-assistance systems have come into pictures. The purpose of surgery robotic-assistant is to provide precise operation with a growing sophisticated instrument. Da Vinci System, a medical robotic system in practice, grants fine surgery in many different therapy targets~\cite{hc_paper_30,hc_paper_32}. For years, clinical trials have been put onto the system. In order to provide reliable healthcare service, safety always ranks first in all considerations~\cite{hc_paper_32}.


Surgery robots' techniques are applied in situ, on treatment, responsible for a medical action.

\item Genome association and gene therapy


Starting from late 20th century, science projects were proposed focusing on various aspects of human genome, such as Human Genome Project, where human DNA sequence was determined, and Gene Ontology, where representation of gene and gene product attributes were unified~\cite{lee2013human},~\cite{gene2007gene}. Recently, genome-wide association study (GWAS) is a popular subdiscipline of genome research, in which hundreds of thousands of single nucleotide polymorphisms (SNPs) are associated with different diseases~\cite{manolio2010genomewide}. Instead of investigating relationships among genes and gene regulators, the innovative GWAS is phenotype oriented, thus breaking the restriction of isolated genes. Consequently, GWAS not only takes advantage of genome research results, but also adaptive to the recently hot topic, big data techniques. By now, GWAS has identified numerous SNPs associated with all kinds of diseases, such as Coronary disease, Childhood asthma and Parkinson’s disease~\cite{visscher201710}. The results from GWAS research provide potential guidance in medical research such as gene therapy development.


\end{itemize}








\subsubsection{Interaction with Other Systems}
%\label{ssec:smart_healthcare_interaction}
\label{sssec:cgh:smart_health:interaction}

Smart healthcare systems are closely related to smart home, ITS and other subsystems as shown in Fig.~\ref{fig:cyx:subsystem-interaction}, while more details will be discussed in section~\ref{sec:InteractionSubsystems}. 



\paragraph{Smart home system}


As discussed in section~\ref{sssec:zcy:smart_home:interaction}, smart home systems need to take into consideration the psychological and physical wellbeing of its users. In this case, the activity detection module of the smart home system can aid in detecting abnormalities of the user's physical wellbeing, especially in senior citizens and children. Through integration with wearable technology, smart home systems can be closer linked with smart healthcare by dynamically adjusting interior environment such as temperature based on the physical and mental state for current users .

\paragraph{ITS: Asuke example}

In many situations when people need help on healthcare, they also need help on transportation. Asuke project, presented in International conference on Universal Village 2016, is an excellent example of taking both need into consideration. 
Focusing on improving the life of elders who live in semi-mountainous areas, Japanese government installs sensors at elderly’s home, in order to watch their health conditions. The sensors send motion data to doctors so that remote diagnosis can be made. Also, this community support system utilizes automated driving technology with fixed route directly to their doctors with comfort. Each car is designed in a small size, which both increases the number of available vehicles and reduces traffic jam. Combined with Intelligent transportation system, Asuke Project provides personalized healthcare services with increasing availability.


% \paragraph{Smart energy management}


\subsubsection{Discussion and Summary}
\label{sssec:cgh:smart_health:discussion}

Current smart healthcare systems are experiencing evolutions with development in computer science and electronic engineering disciplines. New technologies have been examined and applied in an interdisciplinary way, for which paradigms and regulations need to be defined. In a Universal Village circulation, current smart healthcare systems should furthermore consider information and material exchanges with other systems when resolving medical problems, so as not to create additional issues in other systems, such as medical waste disposal, etc.
% current smart healthcare systems should furthermore consider information and material exchanges when resolving medical problems, so as not to create issues in other systems, such as medical waste disposal, medical expenditure so on.


\subsection{Subsystem: ITS, Urban Planning and Crowd management} %\label{ssec:gwy:its_crowd_management}
\label{ssec:gwy:ITS}

Typically, intelligent transportation systems are expected to improve the capacity of urban transport, to improve safety, to reduce energy consumption and to reduce pollutions. 
However, the transportation problems are deeply related to space problems, urban planning and crowd management as discussed in section~\ref{sssec:gwy:ITS:challenges}. 


In this section, we will first analyse current challenges as well as their causes and discuss how ITS should work together with impacting factors to provide an integrated solution. 

\subsubsection{Current Challenges}
%\label{ssec:ITS_challenges}
\label{sssec:gwy:ITS:challenges}

Urbanization promotes enhancement of living standard, extends city scale and stimulates population growth. During the process of urbanization, vehicles are much more affordable for people, whereas price of housing increases rapidly. Therefore, working-class people tend to live in the suburban areas, which leads to longer commuting time than before. To diminish the time consumed by commuting and make schedule flexible, people purchase private cars, which exacerbates loss of public space, traffic congestion, parking difficulties and public transport deficiency~\cite{rodrigue2017geography}. The interactions and causalities between aforementioned problems and other concerned issues, including energy consumption, accidents, pollution, physical and mental health, are shown in Table \ref{tbl:gwy:current_challenges}.

\begin{table}[h!]
	\caption{Current Challenges}
        \begin{center}
                \begin{tabular}{|c|c|c|c|c|}
                        \hline
                                       &   Energy    & \multirow{2}{*}{Accidents} & \multirow{2}{*}{Pollution} & \multirow{2}{*}{Health} \\
                                       & Consumption &                            &                            &                         \\ \hline
                        Loss of public &             &         \checkmark         &         \checkmark         &       \checkmark        \\
                            spaces     &             &            1.a             &            1.a             &           1.a           \\ \hline
                           Traffic     & \checkmark  &         \checkmark         &         \checkmark         &       \checkmark        \\
                          congestion   &     1.b     &            1.b             &            1.b             &           1.b           \\ \hline
                           Parking     & \checkmark  &         \checkmark         &         \checkmark         &       \checkmark        \\
                         difficulties  &     1.c     &            1.c             &            1.c             &           1.c           \\ \hline
                            Public     &             &         \checkmark         &                            &       \checkmark        \\
                          transport    &             &            1.d             &                            &           1.d           \\ \hline
                             Long      & \checkmark  &         \checkmark         &         \checkmark         &       \checkmark        \\
                           commute     &     1.e     &            1.e             &            1.e             &           1.e           \\ \hline
                \end{tabular}
                \label{tbl:gwy:current_challenges}
        \end{center}
\end{table}


\paragraph{Loss of public spaces}

The increase in the number of vehicles has undoubtedly expanded the demand for infrastructure such as roads and parking lots. Available data shows that 30 to 60 percent of the land in the metropolitan area is invested in transportation~\cite{rodrigue2017geography}. The construction of these facilities has led to the loss of public spaces such as parks and green spaces. Public spaces, serving as the city's metabolic system and the public’s leisure and entertainment venues, are continuously being reduced. This not only prohibits the pollutants from being purified but also strengthen the pressure of the public, increasing the risk of depression. In addition, increasingly dense traffic infrastructures and intricate traffic conditions have greatly improved accident rate.


        
\paragraph{Traffic congestion}

Traffic congestion is a common problem in big cities which seriously decreases the usage of public road, makes the driving slow and unsmooth, and increases the chance of having rear-end accidents. Frequent stop-and-gos consume more energy, create more emissions and pollution, and are subject to serious wear-and-tear~\cite{horn2013suppressing}.
% Compared to driving with bilateral control~\cite{horn2013suppressing}, an ideal driving mode proposed at MIT, 
Besides, congestion wears down patience and annoy drivers. Drivers who constantly experience traffic peaks suffer from high stress and negative emotion and are at high risk of mental illnesses such as depression.

\paragraph{Parking difficulties}

Parking difficulties are manifested in two aspects, one is the long time to find a parking space, and the other is a detour in the process of parking. These problems directly lead to more energy consumption and pollutant emissions. Moreover, the \enquote{extra} driving not only creates risks, but also affects the driver's physical and mental health as well as causes diseases, including sedentary-induced obesity and road rage.


\paragraph{Public transport}


The public transportation system is overloaded during peak hours, causing more safety incidents and psychological problems induced by congestion. At certain times, the low occupancy rate of public transportation makes the operation unsustainable and even requires financial subsidies to maintain.


\paragraph{Long commute}


Excessive driving time consumes more energy and produces pollutants. In addition, fatigue driving is a cause of potential safety accidents and is likely to have a negative impact on people's physical and mental health.




\subsubsection{ITS Framework and Functionality}
%\label{ssec:feedbackloop_ITS}
\label{sssec:gwy:ITS:framework}

In view of a Universal Village concept, intelligent transportation systems that fit into the universal subsystem framework should include typical feedback-control loops while the concrete strategies and techniques for all elements are illustrated in Fig.~\ref{fig:gwy:ITS_feedback}. 
The decision-making process is to predict accidents, traffic conditions, crowdedness as well as potential emergencies and assists traffic department in planning future traffic construction and development. The \enquote{action} controls traffic flow and offers dynamic guidance to commuters. \enquote{Data acquisition} detects accidents, collects data of pedestrian behavior and driving behavior, and analyzes these data in combination with other physical statistics and historical data. 
This information is sent throughout the whole loop via cloud storage and communication technologies.



\begin{figure}[h!]
        \centering
        \includegraphics[width=\linewidth]{images/fb_its}
        \caption{Ideal Feedback Control for ITS}
        \label{fig:gwy:ITS_feedback}
\end{figure}


Scholars and engineers from all walks of life have developed corresponding technologies for the functions they concerned. In the next section we will evaluate several typical technologies and how they address current challenges.


\subsubsection{Introduction and Evaluation of Current Technologies}
%\label{ssec:ITS_technology_review}
\label{sssec:gwy:ITS:evaluation}

Nowadays, there are a lot of transportation technologies dedicated to solving some of the aforementioned problems while improving mobility. In this section, we will respectively present the available technologies relevant to four elements, \enquote{Data acquisition \& Communication}, \enquote{Decision making}, and  \enquote{Action},  shown in Fig.~\ref{fig:gwy:ITS_feedback} and then examine them in both their own functional limitations and a Universal Village point of view. 

\begin{table}[h!]
        \caption{Evaluation of current technologies}
        \begin{center}
                \begin{tabular}{|c|c|c|c|c|c|}
                        \hline
                                             &  Technology  &            Energy             &  \multirow{2}{*}{Accidents}   &  \multirow{2}{*}{Pollution}   & \multirow{2}{*}{Health} \\
                                             &  or concept  &          Consumption          &                               &                               &                         \\ \hline
                        \multirow{6}{*}{ITS} &    Smart     & \multirow{3}{*}{$\downarrow$} & \multirow{3}{*}{$\downarrow$} & \multirow{3}{*}{$\downarrow$} &                         \\
                                             &    signal    &                               &                               &                               &                         \\
                                             &   control    &                               &                               &                               &                         \\ \cline{2-6}
                                             &   Electric   & \multirow{2}{*}{$\downarrow$} &                               & \multirow{2}{*}{$\downarrow$} &                         \\
                                             &   vehicles   &                               &                               &                               &                         \\ \cline{2-6}
                                             &  Smart bus   &         $\downarrow$          &                               &         $\downarrow$          &       $\uparrow$        \\ \hline
                        \multirow{2}{*}{UV}  & Proactivity  &                               &         $\downarrow$          &                               &       $\uparrow$        \\ \cline{2-6}
                                             & Event-driven &         $\downarrow$          &         $\downarrow$          &         $\downarrow$          &       $\uparrow$        \\ \hline
                \end{tabular}
                \label{tbl:gwy:uv_lifestyle}
        \end{center}
\end{table}


\paragraph{Data acquisition \& communication}
%Technologies in the element of \enquote{information collection}


As hardware and software technologies advance, more and more methods are available to collect traffic information. A review is available in ~\cite{prabha2016}.


%@article{prabha2016,
%  title={Overview of Data Collection Methods for %Intelligent Transportation Systems},
%  author={Prabha, R. and Kabadi, Mohan G},
%  journal={The International Journal Of Engineering And %Science (IJES)},
%  year={2016},
%  publisher={IEEE},
%}

\begin{itemize}
\item In-situ sensors. Video-based sensors (either visible for daylight or infra-red for night) have been the traditional data collecting methods for ITS. These sensors can provide accurate local condition but with limited coverage and high maintenance cost.

\item Remote sensors. Satellite-based sensors (or photogrammetric technique) can cover a very large area while sacrificing details.  

\item Vehicle-based sensors. Certain types of sensors can be mounted on vehicles and the sensors can communicate with roadside receivers. Modern cell phones have GPS receiver and may be used along with GPS satellites so that locations of vehicles can be tracked in real time. This technique may have privacy concern as the real-time information is collected and passed through communication network. Significant amount of data may be another concern.

\end{itemize}

\paragraph{Decision making}

%\paragraph{Technologies in the element of \enquote{decision making}}

%\begin{itemize}

%\item 

Traditionally, the control center or \enquote{decision making} involves a lot of controllers, and each one is assigned to look at different parts of a city. Manual operations are no longer feasible to monitor transportation patterns in real time while the data quantity gets bigger and bigger.
AI or machine learning and cloud platform are expected to process large amount of data and to help with real-time \enquote{decision making}. At this moment, AI technology still faces many technical challenges, such as, tracking individual targets with complicated backgrounds under all weather conditions, automatically identifying urgent situations, etc. As the computer is getting more powerful, AI will contribute more and alleviate the workload of human traffic controllers.


%As discussed above, when the data quantity gets bigger and bigger, it is no longer feasible to monitor real time transportation patterns though manual operations. The task of \enquote{decision making} thus becomes more challenging as a city gets bigger, more people move in, and the city’s residents get older, which we have discussed above. AI or machine learning and cloud platform may help in terms of processing large amount of data so that human decision makers can render critical decision in real time. Obviously, current AI technology needs to overcome many technical challenges. For examples, how to track a person or a car among many other people or cars under all weather conditions, how to separate emergency from normal condition. Indeed, this paper has some discussion on crowd control system and emergency response system. More details can be found in those sections. As the computer is getting more powerful, AI will contribute more and alleviate the workload of human traffic controllers.

%price(the government aid public transportation to help poor people who la~\cite{stromberg2015real}),transportation convenience(), attitude toward Public Transportation, knowledge~\cite{scott2016public}

%\end{itemize}

\paragraph{Action}
%\paragraph{Technologies in the element of \enquote{action}}

Once we have the information and a decision is made, we need to employ some technologies for the \enquote{action} element to improve the safety and traffic, and thus reducing traffic accidents and pollution. 

First we introduce some safety solutions. According to National Highway Traffic Safety Administration (NHTSA), there are the following ITS safety solutions. 

%https://www.its.dot.gov/factsheets/pdf/ITSA%20ITS%20Saves%20Lives.pdf
\
\begin{itemize}

\item ITS safety solutions \& autonomous vehicles


%\begin{itemize}

%\item 

Vehicle safety systems are the driving aid systems to enhance driving safety and protect drivers in fatigue from making mistakes. 
Typical systems include collision-avoidance systems, lane departure warning systems, drowsy driver warning systems, etc. They are the key technologies developed for autonomous vehicles and start to play an important roles. 

%\item Roadway Safety Systems: Intersection collision avoidance systems, dynamic curve warning systems, wildlife detection systems, road weather sensors

%\item Incident and Emergency Response: Automatic crash notification systems,  Emergency vehicle preemption technology, Real-time data sharing between emergency responders, police, and traffic managers

%\item Prevent crashes before they happen


\end{itemize}


Advanced traffic management systems are expected to assist users in diminishing overall travel time and fuel consumption by avoiding overcrowded routes, which includes the following items. 
Here are some samples.

\begin{itemize}

\item Smart signal control \& traffic management

Signal control is a typical method to control traffic flow and to ensure the orderly and smooth transportation. Smart signal control dynamically adjusts the duration based on road condition and video information, thereby minimizing the throughput time for commuters~\cite{vidhya2014density}. 

\item Electronic billboards 

Electronic billboards may be used to inform drivers where the congestion is so that the drivers may take different routes.

\end{itemize}

\begin{itemize}

\item Electric vehicles

Green-energy vehicles, represented by electric vehicles (EVs), aim to solve the current global energy crisis and improve our environment. According to~\cite{zhou2013development}, it is predicted that EVs will save 55.02\% energy and reduce 40.83\% CO2 emissions in 2020 in China. Currently, EV’s applications are limited by high expense, poor endurance and insufficient charging piles.


\item Smart buses

Smart buses, just like other types of public transportation methods, are expected to help alleviating traffic pressure and protecting the environment. At present, the application of smart bus is not prevalent, and its functions are mainly based on automatic driving and voice control~\cite{lam2014autonomous}. 
Based on the current pilot program, many passengers chose smart buses out of curiosity and had pleasant driving experience and nice mood. 

There are several ITS technologies involved in smart buses. They are  Automatic Vehicle Location System (AVLS), Automated Fare Collection System (AFCS), Traffic Signal Priority (TSP), Driver \& Fuel Monitoring (DFM), and Planning \& Scheduling Software (PSS) etc. Just like other ITS technologies, there are challenges to implement them.

\end{itemize}


\subsubsection{Interaction with Other Systems}
%\label{ssec:ITS_interaction}
\label{sssec:gwy:ITS:interaction}

ITS systems are closely related to smart healthcare systems as discussed in~\ref{sssec:cgh:smart_health:interaction}. 
More details will be discussed in section~\ref{sec:InteractionSubsystems} as shown in Fig.~\ref{fig:cyx:subsystem-interaction}. 

Through analyzing the cause of urban traffic problems and space problems as in section~\ref{sssec:gwy:ITS:challenges}, we notice that the two problems are deeply related to each other. Due to the lack of service awareness in our urban spatial planning, the land is commercialized at the expense of the public space that serves the healthy life of the citizens. 
At the same time, due to the poor sense of sharing, cities have failed to build perfect and convenient public transportation systems, which forced many people to turn to private cars. This in turn leads to space shrinking, parking difficulty, the waste of time and the overdraft of people's emotions, which indirectly lead to the internal consumption of social energy. 
On the other hand, ITS has strong relationship with crowd management. The behavior of human beings are relevant to space usage as well as the transportation. 

From the perspective of Universal Village, ITS should work together with other systems to handle urban space issues, crowd management and transportation together. 
For example, while some highways are very busy during weekdays, they are not fully used in the weekend. Cities can turn a section of such roads into Sunday markets or gathering areas. Besides, the scenic spot and the transportation department can work together in advance to cope with possible traffic problems.

We hope that urban planners fully recognize the relationship between intelligent transportation systems and other subsystems, then adopt different technologies to optimize the city's traffic conditions according to local conditions.


%\paragraph{Emergency dispatch facilitation}


%\paragraph{Disaster response}

The analysis based ITS data can also hint the situations for crowd management system and emergency response system. More details can be found in those sections.

%\paragraph{City planning and event arrangement}

%\paragraph{Tourism management}

%On one hand, smart tourism recommends the tour route based on historical data, and on the other hand, it uses the collected video information to analyze the congestion and crowdedness. Combined with the above two aspects, the scenic spot and the transportation department can prepare in advance to cope with possible traffic problems.

%\paragraph{Safety management}

%\paragraph{Smart mobile community}

\subsubsection{Discussion and Summary}
\label{sssec:gwy:ITS:discussion}

As discussed in section~\ref{ssec:growing_markets}, ITS market is rapidly increasing. As shown In Fig.~\ref{fig:fyj:ITS_market}, ITS market size is expected to reach USD 47.5 billion by 2022, growing at a CAGR of over 13\% from 2015 to 2022. 

NHTSA’s data show that the collision-avoidance systems prevent 17\% of traffic accidents in US and save 17,500 lives, \$26B related costs (vs.\$150B loss)~\cite{kaszynski2000american}. However, considering the 1.35M annual traffic fatalities and 50M injuries worldwide~\cite{fatal-accident}, there is a long way to go in order to achieve the UN goal, decreasing the fatalities and injuries in half, mentioned in section~\ref{ssec:expensive_lifestyle}. The safety goal might not be achieved by ITS only. Collective effort is needed to solve the big challenges. 



%<跳过 smart energy>>>>>>>>>>>>>>>>>>>>>>>>>>>>>




\subsection{Subsystem: Smart Energy Management} \label{ssec:cyx:smart_energy}


\subsubsection{Current Challenges}
\label{sssec:cyx:smart_energy:challenges}

As a result of the rapid growth in world population in recent years, the global energy consumption has increased rapidly~\cite{amasyali2018review}. According to International Energy Agency (IEA), the yearly primary energy supply increased from 8773 Mtoe in 1970 to 13670 Mtoe in 2016, while yearly primary energy consumption increased from 6270 Mtoe in 1970 to 9428 Mtoe Figs.~\ref{fig:cyx:fig1a},~\ref{fig:cyx:fig1b} in 2016~\cite{/content/publication/world_energy_bal-2018-en}. Although the world energy generation generally meets the consumption demand, severe issues exist both locally and globally, such as the uneven distribution of energy resource, continuously increasing greenhouse gas emission, and pollution during energy production. 

\begin{figure}[h!]
        \centering
        \includegraphics[width=\linewidth]{images/cyx_Fig1a}
        \caption{Global Total Energy Supply by Source}
        \label{fig:cyx:fig1a}
\end{figure}

\begin{figure}[h!]
        \centering
        \includegraphics[width=\linewidth]{images/cyx_Fig1b}
        \caption{Global Total Final Consumption by Source}
        \label{fig:cyx:fig1b}
\end{figure}

In some developing countries and regions, energy production significantly relies on imports~\cite{fakharuddin2012smart}. In Taiwan, the electricity demand is 2413 kWh, and it is predicted that the amount will reach 3526 kWh by the year 2030 Fig.~\ref{fig:cyx:fig2}~\cite{energy-2014}, with power resources have gradually failed to meet the demand. The domestic energy pattern has changed significantly in Malaysia, from an energy-rich country to one that lack of indigenous energy resource and relies on imported energy~\cite{fakharuddin2012smart}, resulting in a prospective impact on the domestic economy.

\begin{figure}[h!]
       \centering
       \includegraphics[width=\linewidth]{images/cyx_taiwan_electricity}
       \caption{Electricity Demand Prediction in Taiwan\\
		Note: The blue line refers to the anticipated electricity demand if existing energy-saving measures were continued, with an annual growth of 2.13\% from 2013 to 2030. \\
		The red line refers to the estimated power demand with an annual growth rate of 1.41\% from 2013 to 2030, after implementation of better energy-saving measures.}
		%Source: Energy Bureau\\
		%https://www.slideshare.net/ssuserfc70d8/ss-31844535

       \label{fig:cyx:fig2}
\end{figure}

Fossil fuels are predominant components of today’s energy source: world total primary energy supply (TPES) in 2016 was generated 27\% from coal, 32\% form oil and 22\% from natural gas. While TPES increased by almost 2.5 times between 1971 and 2016~\cite{/content/publication/world_energy_bal-2018-en}, the proportion of fossil fuel as energy source almost remains constant, which reveals remarkable growth of fossil fuel consumption Fig.~\ref{fig:cyx:fig3}~\cite{statistics2017key}. As a result, the greenhouse gas CO\textsubscript{2} emission level was 57.5\% higher in 2016 than in 1990 Fig.~\ref{fig:cyx:fig4}, which is the major contribution to global warming. Therefore, actions must be taken to prevent, or at least alleviate, the process.

\begin{figure}[h!]
        \centering
        \includegraphics[width=\linewidth]{images/cyx_Fig3}
        \caption{Global Total Primary Energy Supply by Fuel}
        \label{fig:cyx:fig3}
\end{figure}

\begin{figure}[h!]
        \centering
        \includegraphics[width=\linewidth]{images/cyx_Fig4}
        \caption{Global Total CO\textsubscript{2} Emission}
        \label{fig:cyx:fig4}
\end{figure}

\subsubsection{Smart Energy Management Systems Framework and Functionality}
\label{sssec:cyx:smart_energy:framework}

Ideal energy management systems take in shape of a typical control flow, as illustrated in Fig.~\ref{fig:cyx:smart_energy_feedback}. Current energy managements techniques can fit into such feedback system accordingly considering their own roles.

\paragraph{Smart Energy Management System}

\begin{figure}[h!]
        \centering
        \includegraphics[width=\linewidth]{images/fb_energy}
        \caption{Ideal Feedback Control for Smart Energy Management System}
        \label{fig:cyx:smart_energy_feedback}
\end{figure}

Energy problems are becoming more crucial to human society due to prospective energy consumption surging, continuing growth in fossil fuel dependency, and the consequent increase in CO2 emission. Researchers are dedicating to addressing these issues by developing energy management systems (EMSs) through building up predictive energy management models~\cite{lee2016energy, wang2017optimal, minoli2017iot, wang2017review}, inventing techniques to efficiently utilize renewable energy~\cite{khare2016solar, fernandez2017new, wang2017toward, feng2007design}, and setting up criteria for evaluating those methods~\cite{roulet2002orme, kumar2017review}. Taking advantage of other subsystems during implementation are also beneficial, such as smart monitoring system for data acquisition and IoT technology for intercommunication.


\paragraph{Renewable Energy Development}

In addition to reducing energy consumption by implementing EMSs, developing renewable energy is an alternative to mitigate the impact by human activity on environment. Although renewable energy only occupies only 12\% of annual TPES, the average increment is 8-9\% per year, which is more than double the average growth of non-renewables. From 2010 to 2017, the world’s total renewable energy installed capacity raised by 78.4\%, from around 1250 kMW to around 2230 kMW, with varying increasing/decreasing rate among energy types Fig.~\ref{fig:cyx:fig5}~\cite{international2017renewables}. 


\begin{figure}[h!]
        \centering
        \includegraphics[width=\linewidth]{images/cyx_Fig5}
        \caption{Global Electricity Generation Trends}
        \label{fig:cyx:fig5}
\end{figure}


\subsubsection{Introduction and Evaluation of Current Technologies}
\label{sssec:cyx:smart_energy:evaluation}

In this section, we will respectively present the available technologies relevant to four elements, \enquote{Data acquisition \& Communication}, \enquote{Decision making}, and  \enquote{Action},  shown in Fig.~\ref{fig:cyx:smart_energy_feedback} and then examine them in both their own functional limitations and a Universal Village point of view.

%\paragraph{Data acquisition and \Communication}
%\begin{itemize}
%\item
%\end{itemize}

\paragraph{Decision making}

\begin{itemize}

\item Data-Driven Predictive Models


While traditional models on energy prediction and management rely on detailed parameters and precise inputs, and are vulnerable to inaccurate simulations, data-driven models, learning from historic data to make predictions, recently caught our eyes~\cite{amasyali2018review}. Studies on data-driven models such as support vector machines (SVMs) and artificial neural networks (ANNs), provide powerful tools and models utilizable on energy prediction and planning. For example,~\cite{candanedo2017data} made a comparison among multiple regression models and developed a non-predictive parameter removal method on energy use of appliances in low-energy houses, and~\cite{ku2015data} presented a Gaussian mixture hidden Markov model to optimize transmission policies for solar-powered sensors. In addition, novel approaches to making higher-level energy consumption predictions are becoming feasible due to the rapid development of big data techniques.~\cite{zhou2016big} developed a power generator planning model by surveying future renewable energy plan combined with predicted gross national product trend. Another advantage of data-driven models over traditional physical models is the adaptability to different data patterns, and the ability to adjust themselves through the feedback loop.

\item Multi Criteria Decision Making

Multi Criteria Decision Making (MCDM) or Multi Criteria Decision Analysis (MCDA) is a branch of Operations Research (OR) that evaluate complex systems with various parameters, objectives and criteria~\cite{kumar2017review}. A typical MDCM procedure consists of a set of systematic steps including system and objective definition, criteria seeking and priority arrangement, serving the purpose of rating and ranking the indicators of the investigated system. With highly diverse appliances and infrastructures, a smart home or building is a complex system consisting of various properties, thus requiring adequately powerful model such as MCDM to analyze it.~\cite{roulet2002orme} is a good example of MCDM model applied on energy management system of office buildings.


\end{itemize}

\paragraph{Action}

\begin{itemize}

\item Energy Saving and Energy Optimization

The main purpose of EMS is to minimize energy cost by optimizing energy usage without dissatisfactory of power demand and discomfort of residents. According to our literature search and a similar survey by~\cite{lee2016energy}, over one hundred cases of building energy management systems (BEMSs) are published in recent decades and distributed all over the world. The research projects include but not limited to system design~\cite{wakui2012optimal},~\cite{olivares2014centralized}, case study~\cite{klein2012coordinating},~\cite{ma2009study}, model development~\cite{sehar2016energy} and algorithm optimization~\cite{shin2016energy}, and scale from single houses~\cite{berlad1976enthalpy} to multiple buildings~\cite{brown1984change}. Various building types are involved, such as residential buildings, commercial buildings and office buildings. While most of these researches focus on energy monitoring and scheduled start/stop of equipment, a potentially smarter way of energy management is to take advantage of IoT technology and novel data-driven methods as described in section A. A good example is described in~\cite{han2010design}, where a smart home EMS is designed and implemented based on ZigBee sensor network~\cite{gascon2009security}, providing adaptive services based on user activity and home environment, and sharing information to users through network. Moreover, optimizing device distributions of other systems is intrinsically minimizing energy consumption, such as trash bins in smart waste management systems and cameras in smart monitoring systems.

\item Blue Energy

With advanced technologies, new types of renewable energy have been discovered, one of which is Blue Energy, which takes advantage of free energy released by mixture of liquid with different salinity~\cite{fernandez2017new}. Current harvest of blue energy relies on triboelectric nanogenerator (TENG), and the related experiments have achieved preliminary results~\cite{xi2017multifunctional}. 
\end{itemize}



\subsubsection{Interaction with Other Systems}
\label{sssec:cyx:smart_energy:interaction}

Energy is a fundamental consideration of all systems, especially infrastructures. Recent popular electrocars and UAVs suffer from the problem of energy shortage and limitation of mobility: batteries usually have worse performance than fossil fuels both in refill speed and endurance. Therefore, decision makers of infrastructure construction need to take energy sustainability into consideration, such as setting up battery recharge/replacement stations.
In addition to routine and time management, smart energy system is also crucial to emergency response systems. Devices in hospitals are directly related to lives, such as lights in surgeries and respirators in ICUs, thus interruption of power supply of those devices are vital in most cases. In emergency situations, emergency response system usually cooperates with transportation system and medical system: emergencies are reported by witnesses, patients are transferred to hospitals by ambulances and then treated by doctors. During those processes, smart energy management system is necessary to ensure the sustainable supply of energy and optimize energy distribution within each system.

\subsubsection{Discussion and Summary}
\label{sssec:cyx:smart_energy:discussion}

As the world energy demand increasing rapidly, it is necessary to both reduce energy waste and develop alternative energy sources by optimizing energy usage and developing renewable energy. In Universal Village concept, we should take advantage of state of art technologies to set up smart energy management systems, as well as discover and improve the efficiency of new energy sources such as blue energy and solar/wind power. Researchers have developed novel methods for energy planning and consumption prediction, including Multi Criteria Decision Making (MCDM) and applied SVMs. In smart cities, energy management system is closely related to other systems, such as infrastructure and emergency response system, by providing basic support and extending mobility.

\subsection{Subsystem: Smart City Infrastructure}
%\label{ssec:City.Infrastructure}
\label{ssec:zlf:smart_infrastructure}

A smart city monitors and integrates all critical infrastructures including water, power, roads, bridges, tunnels, rails, subways, airports, seaports, communications and so on~\cite{hall2000vision}. City infrastructure acts as the foundation of all functional subsystems, and cares about the maintenance and operation for sake of the robust.  

\subsubsection{Current Challenges}
%\label{ssec:infrastructure_challenges}
\label{sssec:zlf:smart_infrastructure:challenges}

Traditionally, infrastructure consists of hard physical assets, for examples roads, airports, hospitals, education facilities, and utility (water, electricity, communication, and sewage) networks. As discussed in ~\cite{mcdonald2018}, the core characteristic of these assets is connectivity, which is consistent with the UV concept of interconnected subsystems. Transportation infrastructure connects people and commercial activities at different physical locations, medical infrastructure connects patients with medical experts, educational infrastructure connects people with different knowledge, and utility infrastructure connects customers with service providers. In summary, traditional infrastructure must be getting smart in order to be the backbone of other subsystems.

Advancement of information and communication technologies (ICT) and IoT makes it possible for hard physical assets to become information assets too. %Future directions for some smart city applications include structural health, transportation systems, etc.

The capability and quality of monitoring infrastructure information are key for seamless function of a city. 
Smart city monitoring systems greatly improve our living conditions but bring challenges of sensing and monitoring at the same time. 

The capabilities of communication and computation resources are typically limited and heavily rely on the availability of power systems. Specifically, battery-powered systems are limited by their energy, computational and storage capability~\cite{du2018sensable}.

\subsubsection{Smart Infrastructure Framework and Functionality}
%\label{ssec:feedbackloop_infrastructure}
\label{sssec:zlf:smart_infrastructure:framework}

In view of a Universal Village concept, intelligent infrastructure systems that fit into the universal subsystem framework should include typical feedback-control loops while the concrete strategies and techniques for all elements are illustrated in Fig.~\ref{fig:zlf:smart_infrastructure_feedback}. Closed feedback loops should be formed for any single component of infrastructure as demonstrated in Fig.~\ref{fig:fyj:UV_feedback_loop}.

\begin{figure}[h!]
        \centering
        \includegraphics[width=\linewidth]{images/fb_infrastructure}
        \caption{Ideal Feedback Control for Smart City Infrastructure System}
        \label{fig:zlf:smart_infrastructure_feedback}%{fig:zlfchart}
\end{figure}


%The ICT infrastructure includes sensing system, communication system and computing system as shown in Fig.~\ref{fig:zlf:smart_infrastructure_feedback}.

\enquote{Data acquisition} is the process of collecting data for different infrastructure systems and analyzes these data in combination with other statistics and historical data. 
This information is sent throughout the whole loop via cloud storage and communication technologies.

The \enquote{decision-making} process is to identify defects, to predict potential emergencies, or to conduct other analyses  and to determine how to respond  in real time after collected data is passed back to the servers in control centers. 

Different \enquote{actions} are carried out to handle different events, such as notifying relevant departments and issuing commands and guidance. Human involvement may be needed for certain events. 

\subsubsection{Introduction and Evaluation of Current Technologies}
%\label{ssec:technology_infrastructure}
\label{sssec:zlf:smart_infrastructure:evaluation}


In this section, we will respectively present the available technologies relevant to four elements, \enquote{Data acquisition \& Communication}, \enquote{Decision making}, and  \enquote{Action},  shown in Fig.~\ref{fig:zlf:smart_infrastructure_feedback} and then examine them in both their own functional limitations and a Universal Village point of view.

\paragraph{Data Acquisition}

%\begin{itemize}

%\item

%\end{itemize}
The functioning of smart infrastructure requires a reliable sensor network and communication systems. A sensor network consists of a set of devices that monitor conditions at different locations for specific purposes. The sensor node should have at least the sensing capability to measure the state of interest and the communication capability to report the measurement. Wireless or wired sensors may be deployed to monitor the state and environment of infrastructure~\cite{Yun2011}. In most cases, sensor nodes are wireless, and they communicate wirelessly, making them easy to deploy and robust to wireline faults~\cite{bensaleh2013review}. For smart city sensing applications, wired sensor nodes can be integrated into the wireless sensor network (WSN). Such wired sensor nodes have better capabilities in communication, computing, and storage than ordinary wireless sensor nodes. However, since most sensor nodes in a network for smart city monitoring are wireless, the network is still considered a WSN. Future directions include sensing, networking and data analysis.

In scenarios where manual detection and maintenance could be inappropriate, robotic agents have been adopted to perform cumbersome or dangerous tasks. In some underground cable systems monitoring, autonomous mobile platform agents have been viewed applications~\cite{robotic_cable_detection}. By specialized functionality and architecture designs, robots engaged monitoring systems emancipate manpower as well as ensure safety~\cite{BAK2004125}, \cite{robotic_cable_detection}. 

Smart city infrastructure typically collect data from different detection levels to make sure system robustness. Varying from scenarios, smart city infrastructure conduct safety monitoring, usage monitoring. Separately, by diagnose structural health, safety monitoring identify critical failures and provide valuable information \cite{farrar2006introduction}, \cite{kim2007health}; usage monitoring describes dynamic system consumption volume \cite{SIANO2014461}, \cite{10.1007/978-3-319-98827-6_9} as well as detect high load which may cause potential failures \cite{JACOB20109}. 

\paragraph{Communication}

The communication system mainly includes wireless infrastructure such as fiber optic channels, Wi-Fi networks, wireless hotspots and kiosks~\cite{chourabi2012understanding}, and service-oriented information systems~	\cite{anthopoulos2010digital}. 

The capabilities of communication and computation resources are typically limited because of the small size of the sensor nodes. 
Besides, the available number of sensor nodes is generally limited, especially if they have to provide high-resolution measurements~\cite{faulkner2011next}.

\begin{figure}[h!]
        \centering
        \includegraphics[width=\linewidth]{images/zlf_chart}
        \caption{The ICT infrastructure}
        \label{fig:zlfchart}
\end{figure}

\paragraph{Decision Making}

Given the complex scenarios in the smart infrastructure system, strategies vary from circumstances. To describe system performance and conditions, researchers and engineers defined different indices, upon which thresholds are selected for corresponding methods. A storm water level is used to describe potential on-site flooding~\cite{Brears2018_1}, while a climate resilience is used to indicate the ability to respond to climate stresses~\cite{Brears2018_2}; these descriptors enable understanding different infrastructure components and help to develop reacting strategies at different levels.  

%\paragraph{Action}

%To avoid system failures, actions should be performed for maintenance either periodically or urgently. 
%Robotic agents have replaced human for inspection and maintenance in many infrastructure components~\cite{doi:10.1002/rob.20295}.   

\subsubsection{Interaction with Other Systems}
%\label{ssec:smart_infrastructure_interaction}
\label{sssec:zlf:smart_infrastructure:interaction}

Smart infrastructure is closely related to city emergency, ITS, smart energy management, smart home, as well as other subsystems as shown in Fig.~\ref{fig:cyx:subsystem-interaction}, while more details will be discussed in section~\ref{sec:InteractionSubsystems}. 

\paragraph{Smart response system}

Smart city infrastructure plays an important part in the smart response system of city emergency for improving the performance of early detection and effective feedback, and then, reducing the loss of disasters. The sensing system of smart city infrastructure can serve as \enquote{data acquisition} element, while the communication system of smart city infrastructure can be an effective feedback element. Fast response of emergency provides more time for response system to utilize more resources and handle with emergency scenario. In some previous research~\cite{fantacci2010novel}, wireless infrastructures have been widely explored to manage all the phases of an emergency situation perform well based on Next Generation Grid principle.

%@article{fantacci2010novel,
%  title={A novel communication infrastructure for emergency management: the In. Sy. Eme. vision},
%  author={Fantacci, Romano and Marabissi, Dania and Tarchi, Daniele},
%  journal={Wireless Communications and Mobile Computing},
 % volume={10},
%  number={12},
%  pages={1672--1681},
%  year={2010},
%  publisher={Wiley Online Library}
%}

\paragraph{Energy management}

As discussed in section~\ref{sssec:zlf:smart_infrastructure:challenges}, the capability and quality of infrastructure monitoring are typically limited by availability of power or battery life. The development of smart infrastructures depends on how to effectively reduce the power consumption for IoT sensors and how to effectively make electricity to be available when needed. 

\paragraph{ITS}

Smart infrastructure is closely related to ITS. For example, whenever there are broken pipes, road defects, power shortage, etc., smart infrastructure systems are expected to quickly pinpoint the problems, to report to control center, to send maintenance vehicles and to fix problems. It is important to ensure the smooth transportation route for these maintenance vehicles. Otherwise, the city transportation would be seriously affected. 

Smart parking is another example which need involvement from city infrastructure and intelligent transportation. The parking information should be consolidated in order to make smart parking feasible. The smart parking lot uses the camera to collect and identify the license plate number and start billing when the vehicle enters the parking lot. When the vehicle leaves the parking lot, the license plate is recognized again and the payment QR code is displayed. Ultrasonic sensors and other equipment are installed at each parking space in the parking lot to detect whether there is a vehicle parked. In this way, the location information of the empty parking space can be provided and guide the drivers when their vehicles enters the parking lot.


\paragraph{Healthcare system}

While hospitals cannot be operated without the support of utility infrastructure, 
in order for patients to see medical experts in hospitals, transportation infrastructure has to function, 

\paragraph{Smart home}

Smart homes are ideal data-feeders into the smart infrastructure system. Because of the importance of homes in people’s lives, they spend a significant portion of time at home. Home is also the most irregular place because other places are often highly specialized, with a routine. The aforementioned conditions make it so that data collected by smart home systems can be of impotance to other systems, especially city-planning, which falls under the jurisdiction of smart infrastructure systems.

\paragraph{Smart response system}

Smart city infrastructure plays an important part in the smart response system of city emergency for improving the performance of early detection and effective feedback, and then, reducing the loss of disasters. The sensing system of smart city infrastructure can serve as \enquote{data acquisition} element, while the communication system of smart city infrastructure can be an effective feedback element. Fast response of emergency provides more time for response system to utilize more resources and handle with emergency scenario. In some previous research~\cite{fantacci2010novel}, wireless infrastructures have been widely explored to manage all the phases of an emergency situation perform well based on Next Generation Grid principle.

%@article{fantacci2010novel,
%  title={A novel communication infrastructure for emergency management: the In. Sy. Eme. vision},
%  author={Fantacci, Romano and Marabissi, Dania and Tarchi, Daniele},
%  journal={Wireless Communications and Mobile Computing},
 % volume={10},
%  number={12},
%  pages={1672--1681},
%  year={2010},
%  publisher={Wiley Online Library}
%}

\paragraph{Energy management}

As discussed in section~\ref{sssec:zlf:smart_infrastructure:challenges}, the capability and quality of infrastructure monitoring are typically limited by availability of power or battery life. The development of smart infrastructures depends on how to effectively reduce the power consumption for IoT sensors and how to effectively make electricity to be available when needed. 

\paragraph{ITS}


Smart infrastructure is closely related to ITS. For example, whenever there are broken pipes, road defects, power shortage, etc., smart infrastructure systems are expected to quickly pinpoint the problems, to report to control center, to send maintenance vehicles and to fix problems. It is important to ensure the smooth transportation route for these maintenance  vehicles. Otherwise, the city transportation would be seriously affected. 


Smart parking is another example which need involvement from city infrastructure and intelligent transportation. The parking information should be consolidated in order to make smart parking feasible. The smart parking lot uses the camera to collect and identify the license plate number and start billing when the vehicle enters the parking lot. When the vehicle leaves the parking lot, the license plate is recognized again and the payment QR code is displayed. Ultrasonic sensors and other equipment are installed at each parking space in the parking lot to detect whether there is a vehicle parked. In this way, the location information of the empty parking space can be provided and guide the drivers when their vehicles enters the parking lot.



\paragraph{Healthcare system}

While hospitals cannot be operated without the support of utility infrastructure, 
in order for patients to see medical experts in hospitals, transportation infrastructure has to function, 


\paragraph{Smart home}

Smart homes are ideal data-feeders into the smart infrastructure system. Because of the importance of homes in people’s lives, they spend a significant portion of time at home. Home is also the most irregular place because other places are often highly specialized, with a routine. The aforementioned conditions make it so that data collected by smart home systems can be of importance to other systems, especially city-planning, which falls under the jurisdiction of smart infrastructure systems.



\subsubsection{Discussion and Summary}
%\label{ssec:discussion_infrastructure}
\label{sssec:zlf:smart_infrastructure:discussion}


As discussed in section~\ref{sssec:zlf:smart_infrastructure:interaction}, city infrastructure subsystems are closely related to many other subsystems for smart cities, especially the smart response systems for city emergency, intelligent transportation, and smart healthcare, whenever there are need for emergency rescue or maintenance needs. The interaction of educational infrastructure and the other types of infrastructure is to provide trained experts to serve the latter.
In order to comprehend the full picture of smart infrastructure, MIMO analysis discussed in section~\ref{ssec:UV.Design} may become necessary. It is clear that multiple feedback loops can be formed, and the loops interact at different spatial and temporal scales. 




\subsection{Subsystem: Smart Response System for City Emergency}
%\label{ssec:City.Emergency}
\label{ssec:zj:smart_emergency}

Smart cities are expected to have the ability to handle unexpected incidents and to reduce the loss of property and human’s life. The efficiency of city emergency response reflects intelligence of a city to a large extent, 
Here below we will introduce city resilience, evaluate related technologies, discuss a case study and explore interactions with other subsystem.


\subsubsection{Current Challenges: Category of Emergency and City Resilience}
%\label{ssec:emergency_challenges}
\label{sssec:zj:smart_emergency:challenges}

City emergency not only results from natural conditions but also some human behaviors, such as high population growth and urbanization~\cite{yang2017using}. 

For one thing, these tragedies cause grave losses of people’s lives and property. For example, California wildfires have exhausted California's fire agency’s annual budget for 2018 of \$442.8 million and now needs an additional \$234 million. 

For another, many response systems still depend on phone calls to detect emergency, such as forest fire, which is certainly not efficient enough. The two factors provoke enthusiasm for smart response system for emergency.

Ideal smart response system for city emergency is expected to handle accidents with early detection, immediate response and effective feedback, and in this way, to minimize the negative effect of disasters.


\subsubsection{Smart Response Systems for Emergency: Framework and Functionality}
%\label{ssec:feedbackloop_emergency}
\label{sssec:zj:smart_emergency:framework}


%In view of a Universal Village concept, smart response systems for city emergency that fit into the universal subsystem framework should include  typical feedback-control loops while the concrete strategies and techniques for all elements as illustrated in Fig.~\ref{fig:zj:smart_emergency_feedback} .
In order to ensure the resilience of city under natural disaster or unexpected urgent situations, smart infrastructure systems need to implement feedback-control framework with concrete strategies and techniques. 

In view of a Universal Village concept, smart response systems for city emergency that fit into the universal subsystem framework should include  typical feedback-control loops while the concrete strategies and techniques for all elements as illustrated in Fig.~\ref{fig:zj:smart_emergency_feedback} in order to ensure the resilience of city under natural disaster or unexpected urgent situations.

\enquote{Data acquisition} element acquires data from different sources such as satellite imagery, wireless sensors and Internet of Things, social media and mobile records and etc. Collected information such as the maximum number of casualties, public opinions, disaster-sites’ situations, on-site rescue situations, would be fed back into decision element. 

\enquote{Communication} element includes different kinds of ways that information about disaster can be timely spread out, such as Wi-Fi, social media and so on. 

\enquote{Decision making} element include models in different domains, such as transportation, electricity and natural disasters, which work together to obtain an accurate approximation of real emergency situations. 


Finally,  based on the evaluation and decision, \enquote{action} elements will take actions responding to different disasters, including human resource allocation, command, dispatching and so on. 

\begin{figure}[h!]
        \centering
        \includegraphics[width=\linewidth]{images/fb_emergency}
        \caption{Ideal Feedback Control for emergency response system}
        \label{fig:zj:smart_emergency_feedback} 
\end{figure}



\subsubsection{Introduction and Evaluation of Current Technologies}
%\label{ssec:technology_emergency}
\label{sssec:zj:smart_emergency:evaluation}

In this section, we will respectively present the available technologies relevant to four elements, \enquote{Data acquisition \& Communication}, \enquote{Decision making}, and  \enquote{Action},  shown in Fig.~\ref{fig:zj:smart_emergency_feedback} and then examine them in both their own functional limitations and a Universal Village point of view.



Big data enables smart systems to operate on vast in-time data~\cite{hashem2015rise}, which is crucial in emergency response, since an effective way of alleviating the effects of disasters is to timely collect the disaster-related data and give timely response like evacuation routes.

Moreover, with the capability of timely data collection, data analysis and data visualization, smart response systems can easily incorporate some external human resource into local environment, enable efficient information exchange between different decision layers. For example, smart response system can track people’s real-time location and electricity loss during disasters through social network. And then, smart response system can provide immediate actions from their action lists. 





\subsubsection{Interaction with Other Systems}
%\label{ssec:emergency_interaction}
\label{sssec:zj:smart_emergency:interaction}

%\subsubsection{Product Design for Responding to City Emergencies}

\paragraph{Smart healthcare: First-aid}

Smart healthcare can contribute to smart response system with immediate medical assistance, such as first-aid products and so on. With the application of various advanced technologies in smart response system, people are paying more and more attention to the research and development of first-aid products, and many design competitions have also added relevant design directions. For example, the German Red Dot Award, which has an international reputation, has opened up a category of awards closely related to emergency products - life sciences and medical technology. And in 2003, the life science and medicine Aids were opened. Since 2001, the American Society of Industrial Designers (IDSA) has added the category Emergency Response Designs to its award-winning international design award, the IDEA award. 

\paragraph{ITS}

Smart emergency systems interact a lot with ITS because ITS can assist smart emergency system with timely data collection and response actions. Reducing the response time of the response system is very effective in mitigating the damage caused by emergency. The information provided by ITS, such as congestions and change of traffic flows, can be incorporated to the decision element for the evaluation of instant impact of emergency. What’s more, this information is helpful for the \enquote{action} element to make a better reaction, like more reliable evacuation routes and more in-time human assistance to the emergency scene.

\subsubsection{Discussion and Summary}
\label{sssec:css:smart_response_system:discussion}
\subsubsection{Case Study: Crowd Resilience System}

%<查到这里>>>>>>>>>>>>>>>>>>>>>>>>>>>>>

Mecca crowd management system can be a typical emergency resilience system because it’s one of the world largest annually mass gatherings. The number of pilgrims gathered in Mecca, a holy city for the Muslins, in 2012 was over 3 million~\cite{alnabulsi2014social}. While majority of them do not speak or understand the Arabic language, it may contribute to the tragedy happened at 2015’s pilgrimage, which claimed the lives of more than 2000 people~\cite{yang2017using}. After that, some researchers designed a crowd management framework with RFID and wireless technologies~\cite{yamin2009crowd}. Similarly, the government of Saudi Arabia designed a more resilient hajj system by setting up 18000 security surveillance cameras, introducing electronic identification bracelets that can help authorities gather individual and crowd information.~\cite{yang2017using}. This information can be used to evaluate the crowd density and flow. The evaluation result can be used for the next decision-making step. The whole system is illustrated in Fig.~\ref{fig:zj:more_resilience}~\cite{yang2017using}.

\begin{figure}[h!]
        \centering
        \includegraphics[width=\linewidth]{images/zj_more_resilience}
        \caption{Case study: Mecca crowd management}
        \label{fig:zj:more_resilience}
\end{figure}

\subsubsection{Product for emergency and UV SDG}

In order to provide more effective, convenient and more humane first-aid products for city emergency, designers should carefully consider various factors before design. 

First, the design must adhere to the \enquote{user-centric} principle, that is, the effectiveness, efficiency, and user satisfaction of the product for a specific user for a specific use in a specific use environment. The extended meaning also includes the ease of learning for a particular user product, the degree of attraction to the user, and the overall psychological feeling of the user before and after experiencing the product, which echos the requirement of  \enquote{inclusive} set by UV SDG as mentioned in~\ref{ssec:UN-development-goals}.

Second, the design should consider the environmental characteristics of the affected area, including the type of disaster, the traits of the disaster site, and the time of the disaster. Among them, the traits of the disaster site (location factor) include geographical features, climate characteristics, historical experience, social culture, and economic characteristics. While the same type of disaster can happen in different environments, and the solution can be very different. Therefore, when designers provide rescue solutions for disaster relief, they need to carefully analyze the environmental characteristics of event development, and provide targeted design solutions according to the type of disaster, the location and time of the disaster. 

Third, the design should consider the needs of different disaster stages. The time of the disaster is relatively limited, but the damage to the victims and the people's response to the disaster are long-lasting. In the early stages of the disaster, during the outbreak and the end, the problems to be solved by the design are also different as shown in Fig.~\ref{fig:css:emergency_rescue_design}.

\begin{figure}[h!]
        \centering
        \includegraphics[width=\linewidth]{images/css_emergency_rescue_design}
        \caption{Value system for emergency rescue design}
        \label{fig:css:emergency_rescue_design}
\end{figure}





\subsection{Subsystem: Smart Environmental Protection}
\label{ssec:zzj:smart_environment}

\subsubsection{Current Challenges}
\label{sssec:zzj:smart_environment:challenges}

Since the emergence of human beings, we have been adapting to the environment, taking advantage of natural resources and transforming our habitats. Among all the human's activities, we have to admit that overexploitation and excessive emissions have fatal damage to the environment. Till now, we have witnessed several serious consequences, sandstorm, soil erosion and desertification. With the development of technology, human live in a modern city and consequently some new environment problems. 

\subsubsection*{Short-term Environmental Problems}


\paragraph{Transportation pollution}

In recent 20 years, car ownership is growing rapidly worldwide, which will rise to 2.5 billion by 2050~\cite{car-population2011}. Moreover, vehicles can emit various pollutants such as carbon monoxide, hydrocarbons, nitrogen oxides, carbon dioxide and particulate matter, which is the main sources of air pollution. Besides, cars are driven by gasoline and oil. The increase number of vehicles means the increase consumption of fossil fuel. In addition, transportation could cause other kinds of pollution --- noise pollution. 

\paragraph{Electronic waste pollution}

When it comes to electronic devices, we are familiar with our smart phones, laptop or iPad, etc. And Waste electric and electronic equipment (WEEE), or electronic waste (e-waste), has been taken into consideration not only by the government but also by the public due to their hazardous material contents~\cite{ec2000draft, cui2003mechanical, epceu2003directive}. Fig.~\ref{fig:zjj:table1}~\cite{hoornweg2012waste} gives examples of the metal composition of different electronic scraps from literatures. 

\begin{figure}[h!]
        \centering
        \includegraphics[width=\linewidth]{images/zjj_table1}
        \caption{Examples of the metal composition of different electronic scraps}
        \label{fig:zjj:table1}
\end{figure}

It shows that if human discard such kinds of electronic devices, their metal contents would have bad influence on the environment, especially the solid and water. According to a report, a battery that contains myriad of metals can pollute 600,000 liters of water~\cite{battery-pollution2002}. Along with the rapid development of technology and urbanization, mankind update laptops quickly, which is to say that the speed of throwing the old-date devices becomes fast and the amount of e-waste increases rapidly. 

\paragraph{Rubbish pollution}

Throwing rubbish seems to be our daily schedule but sometimes we ignore threats of rubbish if we don't preprocess them in advance. In the past century, as the world's population has grown and become more urban and affluent, waste production has risen tenfold. By 2025 it will double again~\cite{spalding1997new}. However, this rubbish contains toxic wastes, such as plastic materials. Durable and very slow to degrade, plastic materials that are used in the production of so many products all, ultimately, become waste with staying power. Published in the journal Science in February 2015, a study conducted by a scientific working group at UC Santa Barbara's National Center for Ecological Analysis and Synthesis (NCEAS) worked out the results: every year, 8 million metric tons of plastic end up in our oceans~\cite{ocean-plastic2010}. It's equivalent to five grocery bags filled with plastic for every foot of coastline in the world. In 2025, the annual input is estimated to be about twice greater, or 10 bags full of plastic per foot of coastline. So, the cumulative input for 2025 would be nearly 20 times the 8 million metric tons estimate \~ 100 bags of plastic per foot of coastline in the world!

%\subsubsection*{Long-term Environmental Problems}

\paragraph{Global warming}

With the accumulation of greenhouse gases such as carbon dioxide, the atmospheric ozone layer is further damaged and solar radiation is increasing, resulting in global warming. Furthermore, the sea level rises gradually until there is no habitat for human and creatures. The whole earth will become a water planet in the end.

\paragraph{Extinction}

Since the environmental pollution and global warming, it is highly possible that lots of animals extinct. For instance, most of crops, vegetables, and fruits rely on bees to pollinate, and even the weeds that we feed animals need to be pollinated by bees. According to statistics, more than 1,000 out of 1,330 crops that can be used by humans rely on bees to pollinate~\cite{worldwatch1998bees}. Moreover, in the past 50 years, crops that require bee's pollination have tripled. However, the number of bees has declined recently, which could result in the shortage of food in the end. Besides, coral reef, often called \enquote{rainforest of the sea}, provides a home for at least 25\% of all marine species~\cite{white2008contaminant}. But this specie is facing a serious problem --- coral bleaching, the main reason of which is global warming. What's worse, the decline of coral reef makes numerous marine creatures lose their homes and eventually extinct.

\paragraph{new types of pollutions}
The process of urbanization has also raised new challenges such as light pollution, noise pollution, even radioactive contamination.
 
Noise pollution is a new type of pollution that has emerged in recent years. According to a research~\cite{pathak2008evaluation}, 85\% of the people were disturbed by traffic noise, about 90\% of the people reported that traffic noise is the main cause of headache, high BP problem, dizziness and fatigue. People having higher education and income level are much aware of the health impact due to traffic noise. And transportation is one of the main resources of noise pollution. For example, the sound of plane taking-off is up to 140db where the standard for the environment noise is 30db. Unfortunately, governments and society don't pay much attention on this problem.
 
Light pollution, as the side effect of the industrial civilization, has disrupted the ecosystem significantly, especially to nocturnal wildlife and animal navigation. Estimates by the U.S. Fish and Wildlife Service of the number of birds killed after being attracted to tall towers range from 4 to 5 million per year to an order of magnitude higher~\cite{Malakoff2001bird}.
 
Light pollution has also influenced the astronomy observations negatively by reducing the contrast between stars and galaxies and the sky itself.
 
Followed by the development of nuclear energy, increasing cases of radioactive contamination have occurred all over the world including Bikini Atoll, the Rocky Flats Plant in Colorado, the Fukushima Daiichi nuclear disaster, the Chernobyl disaster, and the area around the Mayak facility in Russia.
 
However, despite the existing technologies such as noise-canceling headphones for noise pollution or motion sensors for light pollution, these new-born environment problems and challenges are usually lack of attention as well as the effective solutions.

\subsubsection{Framework and Functionality}
\label{sssec:zzj:smart_environment:framework}

A typical feedback system would consist of four elements, including \enquote{decision making}, \enquote{action}, \enquote{data acquisition} and \enquote{communication}, as illustrated in Fig.~\ref{fig:zzj:smart_environment_feedback}. \enquote{Data acquisition} detects pollution, collects data and analyzes data. This information is sent throughout the whole loop via local storage and \enquote{communication} technologies. The \enquote{decision making} process is to make prediction, process data and give instructions when the system finds out pollutions. The \enquote{action} informs the progress of pollution control. 

\begin{figure}[h!]
        \centering
        \includegraphics[width=\linewidth]{images/fb_environment}
        \caption{Ideal Feedback Control for Smart Environmental Protection system}
        \label{fig:zzj:smart_environment_feedback}
\end{figure}


%\subsubsection{Introduction and Evaluation of Current Protection System}
\subsubsection{Introduction and Evaluation of Current Technologies}
\label{sssec:zzj:smart_environment:evaluate}


In this section, we will respectively present the available technologies relevant to four elements, \enquote{Data acquisition \& Communication}, \enquote{Decision making}, and  \enquote{Action},  shown in Fig.~\ref{fig:zzj:smart_environment_feedback} and then examine them in both their own functional limitations and a Universal Village point of view.



\paragraph{Data acquisition and Communication}

\begin{itemize}

\item Unmanned Aerial Vehicle (UAV) monitoring

A UAV is an unmanned aerial vehicle that is controlled by an onboard computer program system or a radio control device. At present, drones have been applied more frequently in the field of environmental protection, such as climate monitoring, water monitoring, air detection and solid protection~\cite{conte2008integrated}. In recent years, the Ministry of Environmental Protection, China has used drones several times to directly inspect the sewage disposal and desulfurization facilities of key enterprises. For example, Wuhan government used drones to track chimneys and detected smoke emission. Heilongjiang government used drones to monitor straw burning. And Lanzhou government used drones to protect winter defenses. 

UAVs are equipped with some sensors. For example, one is aerial image sensor~\cite{quaritsch2010networked}, namely mounting a high-resolution digital camera or video camera on the drone. The aerial images can be transmitted back to our computer in real time, or can be stored and be collected after the drone is landed. Besides, the image mosaic technology can also be used to form an overall cognition of the large-area environment, thereby observing whether there is any smuggling of exhaust gas. Another one is airborne atmospheric environment monitoring sensor~\cite{colomina2014unmanned, klemas2015coastal, turner2012automated}. It can be classified into two devices. One is a spectrum type device based on a two-dimensional planar aerial operation mode, such as a gas filter analyzer, an infrared interferometer, a Fourier transform interferometer and visible light radiation; the other is airborne gas monitoring equipment based on pumping point sampling monitoring mode, such as particle detectors, differential absorption spectrum detection systems, electrochemical gas monitoring equipment.

However, because of the limitation of technology, there are some drawbacks. UAV can detect that there is pollution in this region but cannot locate it precisely. What's more, airflow can interfere with samples during flight which makes UAV cannot follow its regular route. Consequently, human still need to put more efforts on this issue.

\item  Robot monitoring

Robots are mostly be used to monitor a set of serious natural disasters and environmentally harmful accidents~\cite{dunbabin2012robots}. Nowadays, robots can be seen operating in natural or in man-made, highly unstructured environments, such as deep oceans~\cite{whitcomb2010navigation}, active volcanoes~\cite{astuti2009overview}, or damaged nuclear power plants~\cite{guizzo2011japan}. 

In order to monitor a large area, a significant amount of research in multi-agent systems have been dedicated to the development of and experimentation on methods, algorithms and evaluation methodologies for multi-robot patrolling in different scenarios. In~\cite{espina2011multi}, researchers develop and test an integrated multirobot system as a mobile, reconfigurable, multi-camera video-surveillance system. The robots can go around the world, beneath the sand, going into rocks. And then through their mobility, they can collect video data and sensor data in real time.

\end{itemize}

\paragraph{Decision making}

\begin{itemize}

\item Big Data

Big Data is currently used for research and decision-making of environmental science department. United States Environmental Protection Agency (EPA) applies Big Data to the predict mitigation of toxicological issues of industrial chemicals released into the atmosphere. Compared the traditional localized environmental sampling, Big Data allows high throughput, combined data set, and meta-analysis. Big Data is also involved in the uses for Geographic Information Systems(GIS), which enable it to consolidate, utilize and present statistical data for more informed the decision on tracking floodplains and the spread of protected species. Despite the successful applications in climate modelling and species analysis, there are still few areas where Big Data is useful in such areas as land conservation, sustainability and local environmental mitigation, which requires further exploration and research~\cite{bigdata-2018}.

\end{itemize}

\paragraph{Action}

\begin{itemize}

\item Species conservation

Machine learning, image processing, and recognition technologies are applied to species conservation in order to predict changing habitat of certain species, and their migration movement pattern. The Cornell Institute and Cornell Lab of Ornithology have developed software, ebird, which analyzes the data of birds collected from the volunteers for bird habit protection~\cite{AIsave-2018}.
These techniques are also used to identify the endangered species by Google's Cloud AutoML Vision software, which provides immediate data on wildlife populations and distribution to conservation workers, and send alerts and warning to authorities for real-time response~\cite{facialgoogle-2018}.
Further step will be data consolidation by combining the data from all the data sources of different type of species for the analysis of the entire animal ecosystem~\cite{bigprotect-2018}.

\item Robotic decontamination

Besides the application of monitoring, there are also researches focusing on robotic decontamination, which can be applied in processing toxin pollution in extreme environments. International Research Institute for Nuclear Decommissioning (IRID) recently introduced robots to promote decommissioning process in Fukushima Daiichi Nuclear Power Station, expecting to apply remote techniques further to the operations, such as the decontamination of the high-dose radiation area inside the reactor buildings and the removal of fuel debris from the reactor containment vessel or the reactor pressure vessel in order to reduce of operational risk of radiation exposure~\cite{application-robot}. The robotic decontamination, which is costly and time-consuming during the developing process, seems promising in extreme conditions, but requires cost reduction for general and public decontamination purposes.

\end{itemize}

%\subsubsection{Interaction with Other Systems}
\subsubsection{Interaction with other systems}
\label{sssec:zzj:smart_environment:interaction}

\paragraph{Smart Energy Management}

Since the main reason of air pollution comes from fossil fuel, if we can manage energy usability or find clear energy and renew energy, our environment could be more agreeable. As mentioned in section~\ref{ssec:cyx:smart_energy}, Energy Management, there are lots of methods to filter hazardous contents. And renew energy such as solar energy, wind energy and tide energy is applied into our daily life. Besides, according to recent report, China's artificial sun reaches fusion temperature: 100 million degrees, which indicates that if this technology comes true, human don't need to worry about the shortage of energy and greenhouse gas would decrease sharply. 

\paragraph{Interact with IoT}
 
Devices equipped with IoT become more and more popular out of its convenience. Human can monitor environment, get data and control devices in real time. And we are no longer restricted by the distance. Only if there is a electronic device, we can access any sensor and get data, which is necessary for environment protection. If we want to monitor a city's environment, it is convenient for human to detect pollution remotely and in real time. Nowadays, human prefer to install different sensors into UAVs or robots. They can work without rest and can go where people cannot reach. Leveraging devices equipped with sensors, we can save much energy and time to obtain data and monitor pollution through sensors' network.

\subsubsection{Discussion and Summary}
\label{sssec:zzj:smart_environment:discussion}

Current smart environment protection systems mainly focus on traditional pollution detection such as water pollution, green gas and solid pollution. However, there are few technologies or researches on some newborn environmental problem, such as light pollution, noise, even radioactive contamination. Despite the existing technology such as noise canceling headphones  for noise pollution or motion sensor for light pollution, these new-born environment problems are usually lack of effective solutions. In a Universal Village circulation, we cannot emphasize the importance of correlation between current environment protection systems and other systems. When it comes to water pollution, infrastructure systems play a vital role on detecting water contamination via sensor. When we are concerned about air pollution, UVA and Robot can help us collect data. Thus, environment protection systems should furthermore take advantage of information and material transformed from other systems to solve environmental difficulties.
 
In the view of Universal Village merit, besides recognizing  \enquote{awareness of peril} and \enquote{sustainability} of environment, \enquote{diversity and inclusivity} concept requires specific emphasis. Different regions and countries should adopt tailored policies that are in line with their diverse geographic and cultural characteristics in order to increase the feasibility. Meanwhile, the current problem that some individuals intentionally creating information asymmetry for personal-interest by refusing to share or sharing incorrect information should be addressed.




\subsection{Subsystem: Smart Humanity}
\label{ssec:css:smart_humanity}

Whether it is smart home, smart city, smart transportation, smart medical care, when our lives become more and more intelligent because of the development kjof technology, confusions arise: What kind of direction will these technologies bring our lives to? Smart humanity helps people to clarify what the ultimate goal of human development of smart technology is. In the process of exploration, use the smart ideas as a guide to avoid losing the direction of advancement in the carnival of enjoying technological achievements.



\subsubsection{Current Challenges}
\label{sssec:css:smart_humanity:challenges}


\subsubsection{Smart Humanity Framework and Functionality}
\label{sssec:css:smart_humanity:framework}

In view of a Universal Village concept, smart humanity systems that fit into the universal subsystem framework should include typical feedback-control loops while the concrete strategies and techniques for all elements as illustrated in Fig.~\ref{fig:css:smart_humanity_feedback} in order to provide solutions to the problems resulting from the diversity of the society.

\begin{figure}[h!]
        \centering
        \includegraphics[width=\linewidth]{images/fb_humanity}
        \caption{Ideal Feedback Control for Smart Humanity System}
        \label{fig:css:smart_humanity_feedback}
\end{figure}

In order to help the society to achieve the ultimate goal of human development with the smart technology, smart humanity system is designed for individuals with various personal value from different backgrounds to gain the most convenience, security, and other merits in every case. 

\enquote{Data acquisition} element collects the data from individuals, and conveys their’ profile to the society through social media or other public services. Collected information like personal preferences, click rate , or opinion poll reflects the current status of human consciousness which lead to the decision element. 

\enquote{Decision making} element include two main parts. The first is goal setting, like defining the problem, determining the requirements, and etc.. The second is establishing evaluation criteria, which generates a new system based on all the unique personal information. Both methods tend to find the most suitable way to solve daily problems, which lead to the emergence of new products.

The final element \enquote{action} will maximize the merit of either resources of individuals based on the previous evaluation and decision. New products like policies and initiatives will affect people's lifestyles and values without sense aiming for the final human nature harmony. 

\subsubsection{Introduction and Evaluation of Current Technologies}
\label{sssec:css:smart_humanity:evaluation}

In this section, we will respectively present the available technologies relevant to four elements, \enquote{Data acquisition \& Communication}, \enquote{Decision making}, and  \enquote{Action},  shown in Fig.~\ref{fig:css:smart_humanity_feedback} and then examine them in both their own functional limitations and a Universal Village point of view.




\subsubsection{Smart Humanity Influence Problem Discovery and Resolution}
\label{sssec:css:smart_humanity:resolution}

\paragraph{Problem definition and sequencing}

It goes without saying, the application of technology is to solve real-world problems, then what are important issues and which are priorities. For example, smart transportation can not only solve the efficiency problem of urban traffic, avoid the inconvenience caused by traffic jams to individual life, and alleviate the obstacles brought by traffic to the quality of urban operation. But at the same time, smart transportation can also solve the traffic between urban and rural areas, promote the convenience of material transportation, and let the residents of towns and villages enjoy the improvement of their lives in time. In addition, smart transportation can help rural resources to enter the city effectively, promote the development of rural-related industries, and then let the country retain its business and retain people. If the technology is achievable, the definition of these problems and the order in which they are solved depend to a large extent on the values and ideas of people thinking about them, depending on the degree of care of the beneficiaries involved in the application of these technologies.

\paragraph{Different options for the solution}

For the same social problem, smart humanity will influence people's choice of decision-making when solving problems. People consider from different angles, such as whether to consider management convenience first, or user convenience first, which will affect the choice of solutions, thus affecting the direction of smart technology applications. Taking the recycling of urban scrapped private cars as an example, in some developed countries, such as the United States, the automobile recycling industry is relatively complete. They completely disassemble cars that are no longer in use, classify them, and reuse them separately. Even they extracted valuable silicon metal in the exhausted exhaust pipe for important uses. It is a practice of the concept of sustainability, paying more attention to long-term social interests and making the best use of them. On the contrary, in some developing countries, the government will adopt some unified methods, such as letting cars be scrapped after a certain number of years, without considering the reusability of some parts in the car, directly causing waste of resources and causing environmental pollution. This approach is more about considering the convenience of management and focusing on short-term benefits.

\subsubsection{Smart Humanity in Modern Design Culture}
\label{sssec:css:smart_humanity:design}

Different eras have different smart humanities, and these smart humanities are gradually improved in the long-term historical development. Throughout the history of design culture, the \enquote{equality and respect} proposed in the mid-19th century allowed industrial production technology to have consideration of human factors; At the beginning of the 20th century, the Industrial Manufacturing Alliance emphasized \enquote{quality improvement} and promoted Germany as a manufacturing power and technology pioneer. In the mid-20th century, people's attention to \enquote{human rights and responsibility} enabled women to participate in design activities and promote product diversification and humanization. The advocacy of the \enquote{humanitarian spirit} in the second half of the last century promoted the development of universal design. At the beginning of this century, global attention to environmental issues has made sustainable design concepts applicable to all areas of society. The introduction of these wisdom concepts not only lays a foundation for contemporary smart humanity, but also provides a valuable reference for its development.

\subsubsection{Exploration Value System of Contemporary Smart Humanity}
\label{sssec:css:smart_humanity:explore}

\paragraph{Survival and Life: Emphasis on emergency rescue design}

In 2001, Kreps proposed a new definition that disasters are extraordinary events occurring within society or within large social subsystems (such as regions or communities) that result from the combined effects of natural conditions and human society's damage and interference~\cite{kreps2001sociology}. This explanation subverts the passive perception of disasters over the past 40 years. He emphasized that disaster is an abnormal event occurring within society rather than a phenomenon external to society~\cite{sun2014development}. Accountability for human behavior in disaster attribution provokes enthusiasm for disaster prevention. In smart cities, disaster warning systems, disaster response systems (self-rescue product systems, information feedback systems, professional ambulance systems), post-disaster guidance systems (psychological rescue systems, disaster information processing systems, urban construction recovery systems) should be established. The gradual improvement of these systems will enable the public to face all kinds of emergencies in an orderly and effective manner, minimizing possible losses.

\paragraph{Equality and Respect: the continuous upgrading of universal design}

Many scholars interpreted the philosophical basis of universal design concept. D’Souza believes that universal design focuses on the integration of society, giving users the freedom to choose and decide for themselves. Findelli believes that universal design is a \enquote{problem-solving model} in cognitive science, highlighting the initiative of design in solving problems. Tobias pointed out that universal design is a complex of technology, craftsmanship and human actors, so the success of a design cannot be attributed solely to technological innovation. In fact, in our daily lives, many places cannot be visited or many products cannot be used, often not because of the lack of technology, but because of the lack of human consciousness and attitude~\cite{imrie2012universalism}. Taking smart life as an example, with the development of electronic technology and the popularity of electronic products, many stores can only use mobile payment to conduct transactions in order to reflect their smart. This brings convenience to young people while at the same time causing great inconvenience to those who do not have a mobile phone or the aged who are not good at using mobile phones. While pursuing technology, modern life gradually loses the virtue of caring and tolerance for the disadvantaged groups, which is contrary to our pursuit of a better life.

\paragraph{Service and sharing: Improvement of urban space and transportation}

The famous German philosopher and sociologist Jürgen Habermas proposed the theory of communicative action. He believes that the public environment has the functions of understanding, cooperation, socialization and social transformation. Iris Youngr believes that social justice can be realized by creating an inclusive public space that meets the diverse needs of people and fulfills the aspirations of citizens. Taking the material recovery system in a smart city as an example, how to match the user information of \enquote{removal of goods} and \enquote{requirements of old things} in time through the sharing of urban information, design a recycling path that can generate value for each link, and reduce the waste of goods to promotes the beneficial circulation and reuse of goods in the city. The construction of a healthy environment in a smart city requires not only the participation of every citizen, but also the effective integration and output of various information.

\enquote{Service and sharing} is also reflected in the design of urban public transport. At the beginning of this century, many countries in Europe have also taken active measures to improve their public transport services. In 2007, according to the EU survey of 75 cities in Europe, Helsinki, the capital of Finland, had the best public transport. The proportion of public \enquote{special satisfaction} and \enquote{comparative satisfaction} with public transport services reached 48\% and 45\% respectively~\cite{zhao2007quality}. More than that, Finland plans to completely eliminate the need for private vehicles in the city in 2025, starting in Helsinki. In the plan, the government will become a provider of transportation services combining public and private transportation, providing the fastest and cheapest way for citizens to move.

\paragraph{Happiness and health: friendly design of living environment}

A healthy life requires not only healthy food, a pleasant natural environment, but also a friendly and pleasant living environment. A good living environment makes people's lives convenient, relaxed and warm. In a smart city, smart home should provide a comfortable living environment for family members, including the elderly, adults and children. For children, first of all, to ensure the safety of the home environment, such as using hot water reminder device, socket security device. In addition, with regard to the characteristics of modern life, it is possible to allow children and parents to communicate remotely with electronic devices, and to position children in real time to ensure their safety. For adults, it is more reflected in the functional design of the furniture space, such as the smart wardrobe, helping the hostess to quickly choose the clothing suitable for the climate and occasions, so that life becomes convenient and tasteful. Also, the intelligent system of the kitchen is designed to make it possible for the owner to cook. The rational design of the community recycling system allows household waste to be effectively exported in a timely manner. Similarly, smart homes cannot ignore the needs of the elderly, in addition to the special design of the elderly furniture, you can manage pill eating and conduct physical condition tests for the elderly through information interconnection. Let the children at home can pay attention to the daily life of the elderly and provide timely help to their needs, and so on. The rational use of smart technology in the family not only ensures the safety and health of family members, but also promotes the harmony of family members.

\subsubsection{Further Discussion and Summary}
\label{sssec:css:smart_humanity:discussion}

According to the above analysis, an ecological social culture should require a sense of cherishing life, possessing a strong emergency response capability; require a social atmosphere of equality and respect, building a universal environment with strong tolerance; require sharing an efficient urban culture and building efficient and ecological public spaces and public transportation; require a pleasant and healthy living culture, creating a livable living environment suitable for children, the elderly and adults. However, the exploration of a healthy culture requires further exploration, both in breadth and depth. In different social contexts, people have different definitions, choices and applications of health culture. These need to be further developed in future research.

% \subsection{Others}


\section{Interaction of Subsystems and System Dynamics} \label{sec:InteractionSubsystems}

%\subsection{Interaction of Subsystems}
\subsubsection{Interaction with Other Systems}
\label{sssec:css:smart_humanity:interaction}



In previous sections, we have introduced subsystems in smart cities, and dived into each of them to explore state of art technologies as well as research prospective. However, an individual subsystem will be restricted and prevented from powerful and sustainable functioning, although equipped with most advanced technologies, if isolated from other supporting and cooperating systems. Therefore, it is presumptuous to declare and define the perfection of a particular system when neglecting the benefit from others. A typical negative example would be in smart city emergency response system: despite the high density and intelligence of smart monitoring IoT devices, it might not achieve maximum efficiency without the support from intelligent transportation system (ITS) and infrastructure system. In addition, it is noticeable that smart infrastructure system provides most essential platform in order for smart ITS and smart home to operate and optimize.

Strong and direct interactions can be identified between some smart systems, such as aforementioned smart ITS and smart emergency response system, and energy management system and environmental protection system. To be specific, smart ITS has to provide the essential mobile ability and safety assurance to avoid being the bottleneck of emergency response system. On the other hand, smart response system would be beneficial to ITS with excellent performance on rescheduling and resource rearrangement in emergency situations.

Systems involving city and resident routines are also important, especially those that are currently mature in technology and standards and consequently tend to be ignored. With IoT devices installed and managed by smart home systems, residential houses and apartments in future are likely the biggest data resource, both in data size and distribution. Public service systems such as healthcare system, emergency response system and environmental protection system, will potentially be connected to residence buildings and take advantage of those universal data collectors. Data such as human activity, security, energy usage and waste recycling will be collected and analyzed to identify characteristics including lifestyle, safety and environmental properties of local areas. Data related to local traffic routines may also be collected by smart monitoring system in public areas. As a result, stakeholders and decision makers may consider rearranging public resources, such as locations of hospitals, police stations, infrastructure construction plans and electric/water prices. Such shift in system distribution directly contributes to lifestyle properties such as safety, convenience and comfort, and corresponding dependent variables such as humanity and economy, thus in turn changing residency pattern. To summarize, the interaction among individual systems forms a resident-oriented feedback loop of data flow, consisting of all subsystems and their contributions to each other Fig.~\ref{fig:cyx:subsystem-interaction}.

Relatively weak or indirect interactions also exist among subsystems, and a particular subsystem may have to make modifications to adapt to others, such as special processing approaches of medical waste and guaranteed energy availability in emergency situations. It is noticeable that subsystems might interact with each other indirectly by affecting ‘invisible’ factors of city development, such as lifestyle, education and humanity. Those factors are either discussed previously (lifestyle, humanity), or beyond the scope of this paper (education).


\begin{figure}[h!]
        \centering
        \includegraphics[width=\linewidth]{images/cyx_Fig6}
        \caption{Interaction among individual subsystems}
        \label{fig:cyx:subsystem-interaction}
\end{figure}

\renewcommand{\tabularxcolumn}[1]{>{\small}m{#1}}
\renewcommand{\arraystretch}{1.2}

\newgeometry{letterpaper, top=18mm, bottom=18mm, left=18mm, right=18mm}
\begin{landscape}
\begin{table}[p]
        \caption{Interaction of Subsystems}
        \begin{center}
                \begin{tabularx}{1.3\textwidth}{|Y|Y|Y|Y|Y|Y|Y|Y|Y|}
                	\hline
                	                                           & Smart Home \& Community                                                  & Smart Medicine \& Healthcare                  & ITS, Urban Planning \& Crowd Management                              & Smart Energy Management                  & Smart City Infrastructure                           & Smart Response System for City Emergencies                                          & Smart Environmental Protection                         & Smart Humanity                                                       \\ \hline
                	Smart Home \& Community                    &                                                                          & hospitalization: food, care                   & Routine \& Time Management                                           & Routine \& Time Management               & Routine                                             & Home's Disaster/Urgency preparation                                                 & Routine \& Lifestyle                                   & Security System and Data Sharing, Self Sufficiency vs Mutual Support \\ \hline
                	Smart Medicine \& Healthcare               & Activity Detection/Smart Monitoring                                      &                                               & Routine, Pollution, Road Rage                                        & Pollution, Heating/Cooling, Availability & Availability, Lifestyle, Health Status              & First Aid, Health Status                                                            & Pollution, Health Status                               & Universal System, User Friendliness                                  \\ \hline
                	ITS, Urban Planning \& Crowd Management    & Big Event \& Activity, Routine                                           & Location, Limitation of Mobility              &                                                                      & Limitation of Mobility, Lifestyle        & Platform, Fuel Availability, Gas Station Location   & Rescheduling and Resource Rearrangement                                             & Safety, Vehicle Wear and Tear, Lifestyle               & Sharing and Service, Safety, Manner                                  \\ \hline
                	Smart Energy Management                    & Energy Sharing \& Solar Panel                                            & Life Support                                  & Fuel Transportation, Lifestyle, Energy Consumption, Electric Vehicle &                                          & Data Collection, Energy Saving                      & Robustness Test                                                                     & Renewable Energy Development, Energy Saving, Lifestyle & Sustainability, Energy Saving                                        \\ \hline
                	Smart City Infrastructure                  & Smart meter                                                              & Location, Waste Processing, Supporting System & Construction \& Maintainess                                          & Device Life Cycle                        &                                                     & Emergency Awareness, Trial and Robustness Test, Strategic Planning and Optimization & Motivation, Mutual Promotion                           & Fairness, Convenience                                                \\ \hline
                	Smart Response System for City Emergencies & Crowdsource                                                              & Essential Life Support                        & Essential Mobile Need, Safety Assurance                              & Availability, Duration, Connectivity     & Dispatch Planning, Information Collection           &                                                                                     & Prevention and Prediction                              & Prolife, Predictive Design                                           \\ \hline
                	Smart Environmental Protection             & Waste Management \& Recycling                                            & Waste Processing                              & Waste Processing \& Recycling, Pollution Control                     & Pollution, Renewable Energy              & Information Collection, Reduce Pollution, Lifestyle & Improve Awareness                                                                   &                                                        & Human-nature Harmony, Sustainability, Quality of Life                \\ \hline
                	Smart Humanity                             & Safety \& Security, Sense of Belonging, Mutual Support, Harmony, Comfort & Mental Health                                 & Time management, Emotion and Harmony                                 & Survival, Restraint                      & Culture Preservation                                & Calamity Anticipation, Security                                                     & Sense of Responsibility, Envolvement and Cooperation   &                                                                      \\ \hline
                \end{tabularx} 
                \label{tbl:interaction_systems}
        \end{center}
\end{table}
\end{landscape}


\restoregeometry


%\subsection{System Dynamics}

\section{Further Discussion and Summary}
\label{sec:FutureDiscussion}

%\subsubsection{Discussion and Summary}
%\label{sssec:css:smart_humanity:discussion}


As discussed in this paper, the concept of UV has many interconnected subsystems (i.e. basic feedback loops) which interact consistently and affect each other mutually. 

After showcasing eight subsystems and thoroughly discussing their mutual interaction, we summarize and highlight several key concepts and themes that worth people’s attention when they design individual intelligent systems or \enquote{Smart Cities.}

\subsection{Connectivity in four perspectives}
\label{ssec:Connectivity.4.scale}

The seamless information sharing within the UV system is very important for UV mission. Up to now, this paper has discussed connectivity in four perspectives: mutual interaction, feedback loop, information dissemination, and material cycle as summarized below. The paper shows case the above connectivities for eight major intelligent subsystems for Smart Cities. 

First, understanding the connectivity in the layer of mutual interaction is essential to describe complex systems according to the theory of system dynamics. The comprehensive discussion have been presented in section~\ref{sec:InteractionSubsystems} to demonstrate mutual dependence of the eight major systems. It is because that systems in real world are very likely to be non-linear ones. This implies that proper functioning of these systems require broad information sharing across different subsystems and the performance of one system also depends on the ones from serven other systems. When taking into account of mutual interaction, the decision makers would have better understanding the complex systems and can thus discover bottlenecks at strategic level instead of system level only. Optimization should also be based on the total systems instead optimizing eight sub-system independently. Consequently, decision maker can plan in advance by making optimum resource arrangement. Thus super system-wide harmony may be achieved. 

Secondly, as shown in Fig.~\ref{fig:fyj:UV_feedback_loop} in section~\ref{ssec:UV.Design}, the fundamental element of UV system design is feedback loop. Connectivity in the feedback loop is the basic requirement for any intelligent system to function reliably. As illustrated in the discussions for eight seemly-independent subsystems in section~\ref{sec:Subsystems}, in the ideal feedback loops, smart sensors are deployed to collect necessary information for controllers to make certain decision, which is then passed to the actuator to take action. Not all intelligent systems in smart cities can form closed feedback loop. The existence of the physical breakpoints or mismatch of data formats prevent the information from disseminating across every phase of the feedback loop despite that systems have the structure of feedback loop. We identify all relevant technologies involved in three elements and evaluate where closed feedback loops would be formed or not. Such analysis would help decision makers to manage the loop efficiently in two aspects. The first advantage is to help designers to locate breaking points. The second advantage is to assign resources reasonably since the group performance is determined by the bottleneck after recognizing the role and capability of each element. 

Among multiple interconnected systems, a closed loop may not form but necessary information is shared across different systems. An example presented is three modes of connectivity among \enquote{action} element, \enquote{in-situ}, \enquote{in-situ/remote hybrid}, and \enquote{remote}, The first two corresponds to taxi mode / uber mode / reservation for intelligent transportation systems. For intelligent healthcare, such three kind of connectivity correspond to three kinds of doctors-patient relationship, normal doctor / patient meeting at local hospital, Doctor Uber project/medical trip, remote healthcare. There are two types of In-situ hybrid, bringing doctors to patients, (doctor Uber project), and bring patients to doctor, (medical-trip). In particularly, the \enquote{in-situ/remote hybrid} connectivity is an example of adaptive connectivity, which defines dynamic relationship instead of static one. 

The last connectivity is material cycle. Besides typical concept of recycle, UV systems encourage manufacturers to be involved in disassembling their used/waste products into re-usable parts or components before products are scraped and mixed with other metals or materials at recycling centers which are very far away from manufacturing sites. The advantage is to avoid manufacturing the same components from raw materials directly, which minimizes the consumption of natural materials, demands less resource and less transportation need, and agrees with our UV concept.   

This process requires rich know-how about manufacturing parts/components and their physical materials, which can also be treated a closed-loop system. Our proposal on material cycle is designed to achieve harmony between human and its natural environment, just like the previous three perspectives of connectivity. The advantage of information/material connectivity mentioned above can help to coordinate resource allocation which ultimately reduces human’s demand of natural resources, decreases the collective costs. Therefore, there will be less damaging to living environment which ultimately leads to sustainable human happiness in general. 

\subsection{Adaptive and Robust Design for One Complicated System}
\label{ssec:Adaptive.Design}

As mentioned in section~\ref{ssec:jz:intelligent_systems}~\ref{sssec:jz:intsys:adapt}~\ref{sssec:zcy:smart_home:challenges}, adaptivity would be the core feature when designing \enquote{intelligent} system. 
The system design should follow the principle of \enquote{adaptability} and \enquote{robust.} 

The UV framework introduced in section~\ref{sec:UVConcept} treats the whole UV system as one connected complicated system with different subsystems instead of multiple independent system. The mutual interactions among these subsystems would form the dynamic coupling factors.  The mutual interaction naturally leads to additional information connectivities or information loop where the decision signal of one system may be the required input of another system, or vice versa.

Thus traditional static modeling would not work. Instead of having one fixed model who might fail in other situations, taking the dynamic connection into consideration is crucial to ensure \enquote{intelligent} or \enquote{adaptive}. 

\enquote{Robustness} or \enquote{Resilience} is another feature that should be considered when designing systems. We should consider multiple situations when components fail and systems break or are subject to noises. We should also predict and be prepared for the butterfly-effect.  In this paper, we dedicate one subsection to discuss the robustness for cities as  the \enquote{Smart Response System for City Emergency}~\ref{ssec:zj:smart_emergency}.

\paragraph{Proactivity}


Most current artificial intelligence systems are responsive and lack proactivity. For instance, we can use voice message to control smart cars to perform some operations, whereas they probably cannot detect our mood and proactively play appropriate music for us. If the various systems in intelligent transportation are able to take over the system proactively when necessary, then traffic accident rate will be greatly declined.  Commuters who are under pressure also benefit from this feature.


\paragraph{Event-driven}

Event-driven is brought up for situations where some intelligent systems require a large amount of human intervention. Its purpose is to make events the dominant factor in driving operations, thereby making the entire system perform operations more smoothly.


\subsection{UV Themes and Two Subjects}
\label{ssec:Theme}

The objective of Universal Village is to promote \enquote{human-nature} harmony through wise use of technologies. While the majority of the discussion in the paper is about the technologies introduced in section~\ref{sec:UVConcept}, we would like to elaborate on \enquote{human} and \enquote{nature,} the two subjects of UV themes. We wish to ensure the two following key features when designing any subsystems.  
 
\begin{enumerate}
	\item environmental oriented
	\item healthcare \& well being
\end{enumerate}

Thus, healthcare and environment is by no means the task only for medical department and environmental department. Collective efforts are needed to increase the efficiency and responsiveness, reduce demand for natural resources and human resources, intelligently reuse and recycle product parts, thus systematically/ reduce waste, increase people’s happiness and thus meeting the final objective for Universal Village. . 

In summary, our proposed system framework and technology helps to improve some issues  of current technologies for intelligent systems mentioned in section~\ref{sec:SmartTechLimitation} expect the inherent limitations caused by AI algorithms themselves described in section!\ref{ssec:AI.limitation} including the impact of outliers, random activities vs. erroneous profiling, correlation vs. wrong interpretation. An efficient solution needs to combine AI and human beings interdependently, and make them work cooperatively to ensure that UV system be more humanized, more diversified, and more concerned with special situations rather than mechanical judgment and classification.

\section*{Acknowledgment}

We sincerely thank the following people due to their great help for the paper. 

We thank Dr. Ichiro Masaki and Professor Berthold K. P. Horn from MIT UV program for their great contribution in proposing Universal Village concept. 

We thank Professor Lin Zhang, Professor Hongyan Cui, Professor Tao Ma, Dr. Lijuan Su, Dr. Faan Chen and Professor Xiaoman Duan for their wonderful suggestion and support in city evaluation work, respectively in the field of intelligent manufacturing \& system theory, communication \& connectivity, environmental protection, intelligent healthcare, intelligent transportation and energy management. 

We thank Mr. Tian Tan~\cite{uv2018-smartenvironment-tan.tian} who pointed to us the importance of material recycling, which triggered the concept of \enquote{material cycle} mentioned in the paper. 

We thank Yang Liu, Xin Shu~\cite{uv2018-smartenvironment-shu.xin}, Guoxin Huang, Xiushi Wang who initiated the city evaluation work. 

We thank Tian Tan, Longfei Zhou, Fuxin Du~\cite{uv2018-lifestyle-du.fuxin}, Shunzhi Wen~\cite{uv2018-wastemanagement-wen.shunzhi}, Hongji Guo, Fuyuan Zhang, Zhonghe Wang, and Jieke Wang~\cite{uv2018-smartenvironment-wang.jieke} for their research and evaluation work on material cycle, urban lifestyle, and waste management. 

We thank Chuyuan Zhang~\cite{uv2018-smarthome-zhang.chuyuan}, Zhanyuan Huang~\cite{uv2018-smarthome-huang.zhanyuan} and Haosen Cao~\cite{uv2018-smarthome-cao.haosen} for their research and evaluation work on smart home and community. 

We thank Guanghua Chen, Qikai Su~\cite{uv2018-healthcare-su.qikai}, Haoran Ma~\cite{uv2018-healthcare-ma.haoran}, Xiao Yang~\cite{uv2018-healthcare-yang.xiao}, Guoxin Huang~\cite{uv2018-healthcare-huang.guoxin}, and Peize Li~\cite{uv2018-healthcare-li.peize} for their research and evaluation work on smart medicine and healthcare.

We thank Zihan Cao~\cite{uv2018-its-cao.zihan}, Wenyang Gao, Taiyi Wang~\cite{uv2018-its-wang.taiyi}, Shuqing Li and Cairui Wang for their research and evaluation work on intelligent transportation systems, urban planning and crowd management

We thank Yanxi Chen, Jingji Zang~\cite{uv2018-smartenergy-zang.jingji}, Ruiwen Chen and Edgar Fu, for their research work on evaluation work on smart energy management and renewable energy. 

We thank Longfei Zhou~\cite{uv2018-cloudmanufacturing-zhou.longfei}, Guanghua Cheng, Shijun Lun and Ziyan He for their research and evaluation work on smart city infrastructure. 

We thank Jie Zheng and Yuchao Wan~\cite{uv2018-cityemergency-wan.yuchao} for their research and evaluation work on smart response system for city emergency.

We thank Zejun Zhang, Qi Chen~\cite{uv2018-smartenvironment-chen.qi} and Tianyi Zhang for their research and evaluation work on smart environmental protection. 

We thank Shengsheng Cao and Shuqing Li for their research and evaluation work on smart humanity. 

We thank Xiushi Wang~\cite{uv2018-smartdata-wang.xiushi} and Hao Qiu~\cite{uv2018-smartdata-qiu.hao} for their research and evaluation work on data and security for smart cities. 


\printbibliography

\end{document}
